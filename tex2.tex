


\documentclass[UTF8]{ctexart}
 
\usepackage{anyfontsize}  %去除ctex字體報錯
\usepackage{amsmath}
\usepackage{amssymb} 
\usepackage{extarrows}
\usepackage{titlesec}
\usepackage{titletoc}
\usepackage{tikz}
\usetikzlibrary{arrows,backgrounds}
\newcommand{\Rmnum}[1]{\uppercase\expandafter{\romannumeral #1}} 
\newcommand{\mR}[1]{\uppercase\expandafter{\romannumeral #1}} 

\newcommand{\mt}[1]{\text{#1}}
\newcommand{\mb}[1]{\textbf{#1}}
\newcommand{\md}[1]{\displaystyle{#1}}
\newcommand{\mf}[1]{\left( #1\right)}
\newcommand{\mfa}[1]{\left| #1\right|}
\newcommand{\mfb}[1]{\left\{ #1\right\}}
\newcommand{\mfc}[1]{\left[ #1 \right]}
\newcommand{\q}{\quad}
\newcommand{\qa}{\vspace{12 pt}}

\newcommand{\mh}[2]{\overset{#2}{#1}}
\newcommand{\mha}[1]{\overrightarrow{#1}}
\newcommand{\p}{\par}
\newcommand{\ma}[1]{\begin{array}{llll} #1 \end{array}}
\newcommand{\tp}[1]{\begin{tikzpicture}  #1 \end{tikzpicture}}
\newcommand{\tpa}[1]{
    \begin{center}
        \begin{tikzpicture}  
            % [scale=1 ,show background rectangle] 
        
            #1 
            \end{tikzpicture}
    \end{center}
}
\newcommand{\tip}[8]{(intersection of #1,#2--#3,#4 and #5,#6--#7,#8)}

\newcommand{\tpo}[2]{\coordinate  %[label=0:$ {#1} $]
 (#1) at #2; }

\newcommand{\da}[2]{\frac{\partial #1}{\partial #2}}
\newcommand{\db}[2]{\frac{d #1}{d #2}}
\newcommand{\fcz}[1] {
    \left\{
        \begin{array}{llll} #1 \end{array}
    \right.
}

\newcommand{\hls}[1] {
    \left|
        \begin{array}{llll} #1 \end{array}
    \right|
}


\title{This is Start}
\author{Vine}
\date{\today}

\usepackage{geometry}
\geometry{papersize={21cm,29.7cm}}
\geometry{left=1cm,right=1cm,top=2cm,bottom=2cm}

\usepackage{fancyhdr}
\pagestyle{fancy}

\lhead{Vine}
\chead{}
\rhead{}

\lfoot{}
\cfoot{\thepage}
\rfoot{}
\renewcommand{\headrulewidth}{0.1pt}
\renewcommand{\headwidth}{\textwidth}
\renewcommand{\footrulewidth}{0pt}

\usepackage{setspace}
\onehalfspacing
\usepackage{indentfirst}


\begin{document}
\setlength{\headheight}{15pt}


\maketitle %和頁眉衝突


\newpage
\tableofcontents{}





% \newpage

% \section{PCB過孔 (化學沉積) }
% \subsection{材料} 
% \subsubsection{基板}
% \subsubsection{PH調節}
% $HCL,KOH$
% \subsubsection{清洗劑}
% $KOH +$ 洗潔精
% \subsubsection{S溶劑}
% $4\%SnCl_2(PH=2)$
% \subsubsection{P溶劑}
% $0.2PdCl_2(PH=4)$
% \subsubsection{鍍銅液}
% $1.33\%CuSO_4,1.8\%ETDA($乙二胺四乙酸$),10\%$甲醛$(37\%)$
% \subsection{工序} 
% \subsubsection{沉銅} 
% \subsubsection{鍍銅} 


% \section{Today }
% \subsection{C/C++ UI}
% QT,MFC,easyX,javawing
% \subsection{顯影液}
% $H_2SO_4,HNO_3,$苯,甲醇,滷化銀,硼酸,對苯二酚


% \section{2022-04-26 }
% \subsection{WINDOWS API}
% \subsubsection{窗口}
% WinMain,Windows.h,TEXT("abc"),LRESULT
% \subsubsection{question}
% 程序不兼容(SOLVED)
% \subsection{C++ 配置}
% F5,Ctr+Shift+B,-mwindows
% \subsection{launch.json}
% \subsection{task.json}


% \section{2022-04-27 }
% \subsection{makefile}

% \newpage

 
% \newpage
% \section{函數與極限}
% \subsection{映射與函數}

% \subsubsection{映射}
% \textbf{概念}\par
% 非空集合$X,Y$ \quad 法則$f$ \quad $f:X\rightarrow Y$\par
% $D_f=X \quad R_f=f\left(X\right)\subset Y$\par
% $R_f=Y$滿射 \quad $x_1\neq x_2 , f\left(x_1\right)\neq f\left(x_2\right)$單射\par
% 算子 \quad 非空集,數集 (泛函) 
% \quad 非空集,非空集 (變換)
% \quad 實數集/子集,實數集 (函數) \par
% \textbf{逆映射與復合映射}\par
% $g:R_f\rightarrow X$\par
% 對每個$y\in R_f\quad g\left(y\right)=x $ $\quad g$稱為$f$的逆映射,記作$f^{-1}$ \par
% $g:X\rightarrow Y_1,\quad f:Y_2\rightarrow Z \quad Y_1\subset Y_2$\par
% $f\circ g:X\rightarrow Z\quad f\circ g\left(x\right)=g\left[f\left(x\right)\right]\quad x\in X$\par


% \subsubsection{函數}
% \textbf{概念}\par
% 數集$D \in R,\quad f:D\rightarrow R $為定義在D上的函數,簡記$y=f\left(x\right),x\in D$\par
% 自然定義域 $\quad y=|x| \quad y=sgn\left(x\right) \quad y=\left[x\right]
% y=f\left(x\right)\left\{
% \begin{array}{ll}
%     2\sqrt{x},&0<x<=1\\
%     1+x,&1<x\\
% \end{array}
% \right.
% $\par
% \textbf{特性}\par
% 有界性$f\left(x\right)<=K_1 \quad f\left(x\right)>=K_2 \quad |f\left(x\right)|<=M$\par
% 單調性$x_1<x_2,\quad f\left(x_1\right)<f\left(x_2\right)$\par
% 奇偶性$f\left(-x\right)=f\left(x\right) \quad f\left(-x\right)=-f\left(x\right)$\par
% 週期性$f\left(x+l\right)=f\left(x\right)$\par
% 狄利克雷函數 $D\left(x\right)=\left\{ 
%     \begin{array}{l}
%         1,x\in Q\\
%         0,x\in Q^c
%     \end{array}
% \right.$\par
% \textbf{反函數與復合函數}\par
% 函數$f:D \rightarrow f\left(D\right)$是單射\quad 逆映射$f^{-1}:f\left(D\right) \rightarrow D$\quad $f^{-1}$稱為$f$的反函數\par
% $y=f\left(\mu\right),D_f \quad \mu=g\left(x\right),D_g,R_g,R_g\in D_f  \quad y=f\left[g\left(x\right)\right],x\in D_f$
% % \textbf{函數的運算}\par
% % $D\left(x\right)=\left\{ 
% %     \begin{array}{ll}
% %         \left(f\pm g \right)\left(x\right)    &x \in D\\
% %         \left(f . g \right)\left(x\right)    &x \in D\\
% %         \left(\frac{f}{g} \right)\left(x\right)    &x \in D\\
% %     \end{array}
% % \right.$\par

% \textbf{函数的运算}\par
% $f\left(x\right),
% g\left(x\right),
% D=D_f\cap D_g\Rightarrow
% \left\{\begin{array}{ll}
%     \left(f\pm g\right)\left(x\right)=f\left(x\right)\pm g\left(x\right), & x\in D \\
%     \left(f. g\right)\left(x\right)=f\left(x\right). g\left(x\right), & x\in D \\
%     \left(\frac{f}{g}\right)\left(x\right)=\frac{f\left(x\right)}{g\left(x\right)}, & x\in D  \\
% \end{array}\right.$

% \textbf{初等函数}\par
% $\left\{\begin{array}{ll}
%     \text{幂函数}y=x^\mu  &\left(\mu \in R \right)\\
%     \text{指数函数}y=a^x &\left( a>0 \quad a\neq 1 \right)\\
%     \text{对数函数}y=\log_ax &\left( a>0 \quad a\neq 1 ,\quad \log_ex\rightarrow \ln x \right)\\
%     \text{三角函数}y=\sin\left(x\right)  \\
%     \text{反函数}y=\arcsin\left(x\right)  
% \end{array}\right.$\par
% 基本初等函数$\rightarrow$初等函数 \par
% $\left\{\begin{array}{ll}
%     \text{ 双曲正弦}\quad sh\,x =\frac{e^x-e^{-x}}{2}\quad &\text{奇函数}\\
%     \text{ 双曲余弦}\quad ch\,x =\frac{e^x+e^{-x}}{2}\quad &\text{偶函数}\\
%     \text{ 双曲正切}\quad th\,x =\frac{sh\,x}{ch\,x}=\frac{e^x-e^{-x}}{e^x+e^{-x}}\quad &\text{奇函数}\\ 
% \end{array}\right.\Rightarrow
% \left\{\begin{array}{ll}
%     \text{ 反双曲正弦}\quad arsh\,x =\ln \left(x+\sqrt{x_2+1}\right) \quad &\text{奇函数}\\
%     \text{ 反双曲余弦}\quad arch\,x =\ln \left(x+\sqrt{x_2-1}\right) \quad &\text{非奇非偶函数}\\
%     \text{ 反双曲正切}\quad arth\,x =\frac{1}{2}\ln \frac{1+x}{1-x} \quad &\text{奇函数}\\ 
% \end{array}\right.
% $\par








% \subsection{數列的極限}
% $\forall \varepsilon>0,\exists Z^+,\left|x_n-a\right|<\varepsilon \quad \quad \quad \quad \quad \left(n>Z^+\right) $\par
% \subsubsection{定义}
% 法则,  每个$n\in N_+,x_n\rightarrow x_1,x_2,x_3,\dots, x_n,\dots \rightarrow$ 数列$\left\{x_n\right\}$
% \par$x_n$一般项$\left(
%     \frac{n}{n+1},
%     2^n,
%     \frac{1}{2^n}
%     \right)$\par
% $\forall \varepsilon > 0 ,\exists \text{正整数}N,n>N\text{时} , \left|x_n-a \right|<\varepsilon $ 
% \subsubsection{收敛数列的性质}
% \textbf{唯一性}\par
% \textbf{有界性}\quad
% $\left| x_n\right|<=M$\par
% \textbf{保号性}\quad $lim_{n\rightarrow \infty}=a>0\,\left(a<0\right),\exists \text{正整数}N,\text{当}n>N\text{有}x_n>0\,\left(x_n<0\right)$
% \par\textbf{子数列收敛}$\left\{x_n\right\},\left\{x_{n_k}\right\}$


% \subsection{函數的極限}
% % \subsubsection{定義}
% % $\begin{array}{ll}
% %     \textbf{自變量趨於有限值}\\
% %     \forall \varepsilon>0,\exists \delta,\left|f\left(x\right)-a\right|<\varepsilon \quad \,& \left( 0<\left|x-x_0\right|<\delta \right) \\
% %     \textbf{自變量趨於無窮大}\\
% %     \forall \varepsilon>0 ,\exists X>0 ,\left|f\left(x\right)-a \right|<\varepsilon \quad \, & \left(\left|x \right|>X\right)\\
% % \end{array}$\par
% \subsubsection{定义}
% $x_n=f\left(n\right),n\in N^+,n\rightarrow\infty$数列极限
% \par$x\rightarrow x_0,x\rightarrow \infty$函数极限
% \par\textbf{自变量趋于有限值}\par
% 邻域开区间$U\left(x_0,\delta\right),U^\circ\left(x_0,\delta\right)$\par
% $\forall \varepsilon>0,\exists \delta ,\text{当} 0<\left|x-x_0\right|<\delta,\left|f\left(x\right)-a\right|<\varepsilon \Leftrightarrow
% \lim_{x\rightarrow x_0}f\left(x\right)=a
% $\par
% $\forall \varepsilon>0,\exists \delta ,\text{当} x_0-\delta<x<x_0,\left|f\left(x\right)-a\right|<\varepsilon \Leftrightarrow
% \lim_{x\rightarrow x_0^-}f\left(x\right)=a
% $\par
% $\forall \varepsilon>0,\exists \delta ,\text{当} x_0<x<x_0+\delta,\left|f\left(x\right)-a\right|<\varepsilon \Leftrightarrow
% \lim_{x\rightarrow x_0^+}f\left(x\right)=a
% $
% \par\textbf{自变量趋于无穷大}\par
% $\forall \varepsilon>0,\exists \delta ,\text{当} \left|x \right|>\delta,\left|f\left(x\right)-a\right|<\varepsilon \Leftrightarrow
% \lim_{x\rightarrow \infty}f\left(x\right)=a
% $\par

% \subsubsection{性质}
% \textbf{唯一性}\par
% \textbf{局部有界}\par$\lim_{x\rightarrow x_0}f\left(x\right)=a\Rightarrow
% \exists M>0,\delta>0,0<\left|x-x_0\right|<\delta\text{时},\left|f\left(x\right) \right|<=M
% $\par
% \textbf{局部保号}\par
% $\lim_{x\rightarrow x_0}f\left(x\right)=a>0\Rightarrow
% \exists  \delta>0,0<\left|x-x_0\right|<\delta\text{时}, f\left(x\right)>0
% $\par
% $\lim_{x\rightarrow x_0}f\left(x\right)=a\neq 0\Rightarrow
% \exists  \delta>0,0<\left|x-x_0\right|<\delta\text{时}, \left|f\left(x\right)\right|>\frac{\left|a\right|}{2}
% $\par 
% $\exists  \delta>0,0<\left|x-x_0\right|<\delta\text{时} f\left(x\right)>=0, \text{且}\lim_{x\rightarrow x_0}f\left(x\right)=a  
%  \Rightarrow  a>=0
% $\par
% \textbf{函数极限与数列极限}\par
% $\lim_{x\rightarrow x_0}f\left(x\right)=a\Leftrightarrow
% \text{所有的}\lim_{n\rightarrow \infty }x_n=x_0,\text{有}\lim_{n\rightarrow \infty }f\left(x_n\right)=a
% $


% \subsection{無窮小與無窮大}
% \subsubsection{无穷小}
% $\lim_{x\rightarrow x_0}f\left(x\right)=0\Rightarrow f\left(x\right)\text{为} x\rightarrow x_0 \text{时的无穷小}$
% \par$\lim_{x\rightarrow x_0}f\left(x\right)=A \Rightarrow f\left(x\right)=A+\alpha$

% \subsubsection{无穷大}
% $\lim_{x\rightarrow x_0}\left|f\left(x\right)\right|>\text{任意正数} M\Rightarrow f\left(x\right)\text{为} x\rightarrow x_0 \text{时的无穷大}$
% 记为$\lim_{x\rightarrow x_0}f\left(x\right)=\infty$
% \par$\lim_{x\rightarrow x_0}f\left(x\right)=0\Rightarrow \lim_{x\rightarrow x_0}\frac{1}{f\left(x\right)}=\infty $


% \subsection{極限運算法則}

% \subsubsection{兩個無窮小的和}
% 有限個無窮小的和
% \subsubsection{有界函數與無窮小的乘積}
% 常數與無窮小的乘積\par
% 有限個無窮小的乘積
% \subsubsection{$\lim f\left(x\right)=A,\lim g\left(x\right)=B$}
% $\lim\left[f\left(x\right)\pm g\left(x\right)\right]=\lim f\left(x\right)\pm \lim g\left(x\right)=A\pm B$\par
% $\lim\left[f\left(x\right)\cdot g\left(x\right)\right]=\lim f\left(x\right)\cdot \lim g\left(x\right)=A\cdot B$\par
% $\lim\frac{f\left(x\right)}{g\left(x\right)}=\frac{\lim f\left(x\right)}{\lim g\left(x\right)}=\frac{A}{B}\quad \left(B \neq 0 \right)$\par
% $\lim\left[cf\left(x\right)\right]=c\lim\left(x\right)$\par
% $\lim\left[x\right]^n=\left[f\left(x\right)\right]^n$\par

% \subsubsection{數列$\left\{x_n\right\},\left\{y_n\right\},\lim_{n \rightarrow  \infty}x_n=A,\lim_{n \rightarrow  \infty}y_n=B$}
% $\lim_{n \rightarrow  \infty}\left(x_n\pm y_n\right)=A\pm B$\par
% $\lim_{n \rightarrow  \infty}\left(x_n \cdot y_n\right)=A \cdot B$\par
% $\lim_{n \rightarrow  \infty}\frac{x_n}{y_n}=\frac{A}{B}$\par

% \subsubsection{$\displaystyle{\psi  \left(x\right)>=\Psi\left(x\right),
% \lim_{n \rightarrow  \infty}\psi  \left(x\right)=A,
% \lim_{n \rightarrow  \infty}\Psi\left(x\right)=B\Rightarrow A>=B
% }$}
% \subsubsection{復合函數極限}
% $\displaystyle{ 
%     \lim_{x \rightarrow  x_0}g\left(x\right)=\mu_0,
%     \lim_{\mu \rightarrow  \mu_0}f\left(\mu\right)=A,\exists    
% \delta_0>0,x\in \mathring{U}\left(x_0,\delta_0\right)},
% g\left(x\right)\neq \mu_0 \Rightarrow 
% \lim_{x \rightarrow x_0}f\left[g\left(x\right)\right]=
% \lim_{\mu \rightarrow \mu_0}f\left(\mu\right)=A
% $
% % \subsection{極限存在準則}
% \subsection{极限存在准则$\quad$两个重要极限}
% \Rmnum{1}
% 数列$\left\{x_n\right\},
% \left\{y_n\right\},
% \left\{z_n\right\}$
% 满足\par
% $\begin{array}{l}
%     \exists n_0\in \mathbf{N}^+,n>n_0\text{时},y_n\leqslant x_n\leqslant z_n\\
%     \displaystyle{\lim_{n\rightarrow \infty}y_n=a,
%     \lim_{n\rightarrow \infty}x_n=a}
% \end{array}
% \Rightarrow \left\{x_n\right\}$极限存在,且
% $\displaystyle{\lim_{n\rightarrow \infty}x_n=a}
% $\par

% $
% \begin{array}{l}
%     x\in \mathring{U}\left(x_0,r\right)\text{时},g\left( x\right)\leqslant  f\left( x\right)\leqslant  h\left( x\right)\\
%     \displaystyle{\lim_{x\rightarrow x_0}g\left(x\right)=a,
%     \lim_{x\rightarrow x_0}h\left(x\right)=a}
% \end{array}\Rightarrow \text{极限}\displaystyle{\lim_{x\rightarrow x_0}f\left(x\right)\text{存在,且等于}a}
% $\par
% \Rmnum{2}单调有界数列必有极限\par
% $x_1 \leqslant x_2 \leqslant x_3 \leqslant \dots \leqslant x_n \leqslant x_{n+1} \leqslant \dots$
% 单调递增数列\par
% $x_0\text{的某个左邻域内},f\left(x\right)\text{单调有界,则}f\left(x_0^-\right)\text{必定存在}$\par
% 柯西极限存在准则$\quad \left\{x_n\right\}\text{收敛}\Leftrightarrow \forall \varepsilon>0 ,\exists N^+ ,\text{当}m>N,n>N\text{时},\left|x_n-x_m\right|<\varepsilon$

% \subsection{無窮小的比較}
% $\begin{array}{ll}
%     \lim\frac{\beta}{\alpha}=0 & \beta \text{是}\alpha \text{的高階無窮小,記為}\beta=o\left(\alpha\right)\\
%     \lim\frac{\beta}{\alpha}=\infty  &\beta \text{是}\alpha \text{的低階無窮小} \\
%     \lim\frac{\beta}{\alpha}=c\neq 0  &\beta \text{是}\alpha \text{的同階無窮小} \\
%     \lim\frac{\beta}{\alpha^k}=c\neq 0  & \beta \text{是}\alpha \text{的k階無窮小}\\
%     \lim\frac{\beta}{\alpha}=1  & \beta \text{是}\alpha \text{的等價無窮小,記為}\alpha \sim \beta \\
% \end{array}
% $\par
% $\alpha , \beta \text{是等價無窮小}\Leftrightarrow \beta=\alpha +o\left(\alpha \right)$\par
% $\alpha \sim \widetilde{\alpha} ,\beta \sim \widetilde{\beta},\exists \lim \frac{\widetilde{\alpha}}{\widetilde{\beta}}\Rightarrow\lim \frac{\alpha}{\beta}=\lim \frac{\widetilde{\alpha}}{\widetilde{\beta}}$

% \subsection{ 函數的連續性與間斷點}
% \subsubsection{ 函數的連續性}
% $\Delta\mu =\mu_2-\mu_1,x\in U\left(x_0 \right),\Delta y=f\left(x_0+\Delta x\right)-f\left(x_0\right)$\par
% $\ma{
%     f\left(x\right),x \in U\left(x_0\right),\displaystyle{\lim_{\Delta x\rightarrow 0}\Delta y=\lim_{\Delta x\rightarrow 0}\left[ f\left(x_0+\Delta x\right)-f\left(x_0\right)\right]=0} &,\mt{稱}f\left(x\right)\mt{在}x_0\mt{連續}
%     \\ f \mf{x},x \in U \mf{x_0}, \md{\lim_{x\rightarrow x_0}}f \mf{x}=f \mf{x_0} &,\mt{稱} f\mf{x}\mt{在}x_0 \mt{連續}
%     \\ \forall \varepsilon>0,\exists \delta>0 ,\mt{當} \mfa{x-x_0}<\delta \mt{時,有} \mfa{f\mf{x}-f\mf{x_0}}<\varepsilon  &,\mt{稱} f\mf{x}\mt{在}x_0 \mt{連續}
% }$\par
% $f\mf{x_0^-}=f\mf{x_0},\mt{左連續}\quad f\mf{x_0^+}=f\mf{x_0},\mt{右連續}$\par
% 區間上每一點都連續的函數,函數在該區間上連續,包含端點時右端點左連續,左端點右連續\par
% $\left\{\ma{
%     \mt{有理整函數}\mf{\mt{多項式}},\\
%     \mt{有理分式函數},F\mf{x}=\frac{P\mf{x}}{ Q\mf{x}} \\
%     \mt{冪函數},\sqrt{x}\\
%     \mt{三角函數},\sin\,x
% }\right.$

% \subsubsection{ 函數的間斷點}
% $f\mf{x},x\in \mathring{U}\mf{x_0}\left\{
%     \ma{
%         \mf{1}\, x=x_0\mt{處沒定義}\\
%         \mf{2}\, x=x_0\mt{處有定義,但}\md{\lim_{x\rightarrow x_0}f\mf{x}}\mt{不存在} \\
%         \mf{3}\, x=x_0\mt{處有定義,且}\md{\lim_{x\rightarrow x_0}f\mf{x}}\mt{存在,但} \md{\lim_{x\rightarrow x_0}f\mf{x} \neq f\mf{x_0}}
%     }   
% \right.$\par
% 左右極限存在,第一類間斷點$\mf{\mt{可去,跳躍}}$\par
% 不是第一類間斷點,第二類間斷點$\mf{\mt{無窮,震蕩}}$

% \subsection{ 連續函數的運算與初等函數的連續性}

% \subsubsection{ 連續函數的和,差,積,商的連續性}
% $\mt{函數}f\mf{x},g\mf{x}\mt{在}x_0\mt{連續}\Rightarrow f\pm g,f\cdot g,\frac{f}{g}\mt{在點}x_0\mt{連續}$
% \subsubsection{ 反函數與復合函數    的連續性}
% $f\mf{x}\mt{在} D_f\mt{單增},f^-\mf{x} \mt{在} R_f \mt{單增}$\par
% $\left.\ma{y=f\mf{\mu},\mu=g\mf{x}\\
% y=f \mfc{g\mf{x}},\mathring{U}\mf{x_0}\in D_{f\circ g}\\
% \md{\lim_{x\rightarrow x_0}g\mf{x}=\mu_0,f\mf{\mu}\mt{在點}\mu=\mu_0\mt{處連續}}
% }\right\}\Rightarrow\md{\lim_{x \rightarrow x_0}f\mfc{g\mf{x}}=\lim_{\mu\rightarrow \mu_0}f\mf{\mu}=f\mf{\mu_0}}$\par

% $\left.\ma{y=f\mf{\mu},\mu=g\mf{x}\\
% y=f \mfc{g\mf{x}},\mathring{U}\mf{x_0}\in D_{f\circ g}\\
% \mu=g\mf{x}\mt{在點}x=x_0\mt{處連續,且}g\mf{x_0}=\mu_0 \\
% f\mf{\mu}\mt{在點}\mu=\mu_0\mt{處連續}
% }\right\}\Rightarrow f\mfc{g\mf{x}} \mt{在點}x=x_0\mt{處連續} $

% \subsubsection{ 初等函數的連續性}
% 基本初等函數定義域內連續\par
% 初等函數定義區間(定義域內的區間)連續

% \subsection{ 閉區間上連續函數的性質}
% \subsubsection{ 最值定理}
% $f\mf{x}\mt{在區間}\mfc{a,b}\mt{連續},\exists M>0,\forall x \in \mfc{a,b},\mfa{f\mf{x}}\leqslant M$
% \subsubsection{ 零點定理與介值定理}
% $f\mf{x_0}=0,x_0\mt{稱為函數}f\mf{x}\mt{的零點}$\par
% $f\mf{x}\mt{在區間}\mfc{a,b}\mt{連續},f\mf{a}\cdot f\mf{b}>0,\exists \mt{至少一點} \xi \in \mf{a,b},\mt{使得}f\mf{\xi}=0$\par
% $f\mf{x}\mt{在區間}\mfc{a,b}\mt{連續},f\mf{x}=A,f\mf{b}=B,\forall C \in \mf{A,B},\exists \mt{至少一個}\xi \in \mf{a,b},\mt{使得}f\mf{\xi}=C $\par
% $f\mf{x}\mt{在區間}\mfc{a,b}\mt{連續},R_f=\mfc{m,M},m=f\mf{x}_{min},M=f\mf{x}_{max}$

% \subsubsection{ 一致连续性}
% $x\in R_f, \forall \varepsilon,\exists \delta ,\mt{使得}\forall x_1,x_2\in R_f ,\mt{当}\mfa{x_1-x_2}<\delta\mt{时},\mfa{f\mf{x_1}-f\mf{x_2}}<\delta$
% \par 如果函数在$f\mf{x}$在闭区间$\mfc{a,b}$连续,那么他在该区间一定连续



% \newpage
% \section{導數與微分}

% \subsection{導數概念}
% \subsubsection{引例}
% \mb{直線運動的速度}\par
% $s=f\mf{t}$\par
% $\frac{s-s_0}{t-t_0}=\frac{f\mf{t}- f\mf{t_0}}{t-t_0}$平均速度\par
% $v=\md{\lim_{t \rightarrow t_0}\frac{f\mf{t}- f\mf{t_0}}{t-t_0}}$瞬時速度\par
% \mb{切線問題}\par
% $\tan \varphi=\frac{y-y_0}{x-x_0}=\frac{f\mf{x}-f\mf{x_0}}{x-x_0}$割線斜率\par
% $k=\md{\lim_{x \rightarrow x_0}\frac{f\mf{x}-f\mf{x_0}}{x-x_0}} $切線斜率


% \subsubsection{導數的定義}
% \mb{函數在一點處的導數與導函數}\par
% $\md{\lim_{x \rightarrow x_0}\frac{f\mf{x}-f\mf{x_0}}{x-x_0}}$\par
% $\md{\lim_{\Delta x \rightarrow 0}\frac{\Delta y}{\Delta x},\lim_{\Delta x \rightarrow 0}\frac{f\mf{x_0+\Delta x}-f\mf{x_0}}{\Delta x}}$\par
% $\md{\ma{
%     f^{'}\mf{x_0} & =\lim_{\Delta x \rightarrow 0}\frac{f\mf{x_0+\Delta x}-f\mf{x_0}}{\Delta x}\\
%     & =y^{'}|_{x=x_0}\\
%     & =\frac{d\,y}{d\,x}|_{x=x_0}\\
%     & =\frac{d\,f\mf{x}}{d\,x}|_{x=x_0}
% }}$\par
% $\Delta x \rightarrow 0 \mt{時},\frac{\Delta y}{\Delta x} \rightarrow \infty \Rightarrow \mt{不可導}$\par
% $y_{'}=\lim_{\Delta x \rightarrow 0}\frac{f\mf{x+\Delta x}-f\mf{x}}{\Delta x},f^{'}\mf{x},\frac{d\,y}{d\,x},\frac{d\,f\mf{x}}{d\,x}$導函數\par
% $f^{'}\mf{x_0}=f^{'}\mf{x}|_{x=x_0}$

% \mb{求導舉例}\par
% \subsubsection{導數的幾何意義}
% $
% \ma{
%     f^{'}\mf{x_0}=\infty ,x=x_0\\
%     y-y_0=f^{'}\mf{x_0}\mf{x-x_0}\mt{切線}\\
%     y-y_0=-\frac{1}{f^{'}\mf{x_0}}\mf{x-x_0}\mt{法線}
% }
% $

% \subsubsection{可導與連續性關係}
% $\md{\lim_{\Delta x \rightarrow 0}\frac{\Delta y}{\Delta x}=f^{'}\mf{x}\Rightarrow \frac{\Delta y}{\Delta x}=f^{'}\mf{x}+\alpha 
% \Rightarrow  \Delta y=f^{'}\mf{x}\Delta x + \alpha\Delta x
% \Rightarrow \mt{可導必連續}
% }$


% \subsection{函數求導法則}
% \subsubsection{函數和差積商的求導法則}
% $u^{'}\mf{x},v^{'}\mf{x}\Rightarrow
% \ma{
%     \mf{u\pm v}=u^{'} \pm v^{'}\\
%     \mf{uv}^{'}=u^{'}v+uv^{'},\mf{cu}^{'}=cu^{'},\mf{uvw}^{'}=u^{'}vw+uv^{'}w+uvw^{'}\\
%     \mf{\frac{u}{v}}^{'}=\frac{u^{'}v-uv^{'}}{v^2}}$

% \subsubsection{反函數求導法則}
% $x=f\mf{y},f^{'}\mf{y} \neq 0 \rightarrow y=f^{-1}\mf{x}\mt{可導},\mfc{f^{-1}\mf{x}}^{'}=\frac{1}{f^{'}\mf{y}},\frac{dy}{dx}=\frac{1}{\frac{dx}{dy}}$

% \subsubsection{復合函數求導法則}
% $u=g\mf{x}\, x \mt{點可導},y=f\mf{u} \,u\mt{點可導} \rightarrow y=f\mfc{g\mf{x}}\,x \mt{點可導},\frac{dy}{dx}=f^{'}\mf{u}\cdot g^{'}\mf{x},\frac{dy}{dx}=\frac{dy}{du}\cdot\frac{du}{dx}$


% \subsubsection{基本求導法則與導數公式}

% \mb{基本初等函數函數求導公式}\par
% \mb{和差商積求導法則}\par
% \mb{反函數求導法則}\par
% \mb{復合函數求導法則}\par

% \subsection{高階導數}
% $\frac{d}{dx}\mf{\frac{dy}{dx}},\frac{d^2y}{dx^2},y^{''}\mf{x}$\par
% $y^{'''},y^{\mf{4}},\dots,y^{\mf{n}} \rightarrow  \frac{d^3y}{dx^3},\frac{d^4y}{dx^4},\dots,\frac{d^ny}{dx^n} $\par
% $\md{\mf{u+v}^n=\sum_{k=0}^nC_n^ku^{\mf{n-k}}v^{\mf{k}}}$


% \subsection{隱函數,參數方程確定的函數的導數 $\quad$ 相關變化率}
% \mb{隱函數導數}\par
% $y=\sin \, x \quad \mt{顯函數}$\par
% $x+y^3-1=0 \quad \mt{隱函數}, \quad y=\sqrt[3]{1-x} \quad \mt{隱函數顯化}$\par

% \mb{參數方程確定的函數的導數}\par
% $\left\{ \ma{
%     x=\varphi \mf{t}\\
%     y=\psi \mf{t}
% }\right. \rightarrow \md{\frac{dy}{dx}=\frac{\psi^{'} \mf{t}}{\varphi^{'} \mf{t}}}$


% \mb{相關變化率}\par
% $\frac{dy}{dt},\frac{dx}{dt} \mt{存在關係}$


% \subsection{函數的微分}

% \mb{微分的定義}\par
% $\Delta A=2x_0\Delta x + \mf{\Delta x}^2$\par
% $x\in U\mf{x_0}, \Delta y = A \Delta x + o\mf{\Delta x} \rightarrow y\mt{在}x_0 \mt{點可微},A\Delta x \mt{是} y\mt{在}x_0 \mt{點相應于} \Delta x \mt{的微分記為}dy$\par
% $y\mt{在任意點} x \mt{的微分,函數的微分,記作}dy$\par
% $\Delta x \mt{自變量的微分,記作}dx$

% \mb{微分的幾何意義}\par
% \mb{初等函數的微分公式,與微分運算法則}\par
% \begin{enumerate}
%     \item 基本初等
%     \item 和差商積
%     \item 復合函數
% \end{enumerate}

% \mb{微分在近似計算中的應用}\par
% \mb{函數的近似運算}\par
% $\Delta y \approx dy=f^{'}\mf{x_0}\Delta x $

% \mb{誤差估計}\par
% $A\mt{精確值} ,a\mt{測量值},\mfa{A-a}\mt{絕對誤差},\frac{\mfa{A-a}}{\mfa{a}}\mt{相對誤差},\mfa{A-a}\leqslant \delta_A \mt{絕對誤差極限},\frac{\delta_A}{\mfa{a}}\mt{相對誤差極限} $

% \newpage
% \section{微分中值定理與導數的應用}
% \subsection{微分中值定理}

% \mb{費馬引裡} $x \in U\mf{x_0}, f^{'}\mf{x_0},\forall x\in U\mf{x_0},f\mf{x}\leqslant f\mf{x_0} or f\mf{x}\geqslant f\mf{x_0}\rightarrow f^{'}\mf{x_0}=0 \mt{駐點,穩定點,臨界點}$\par

% \mb{儸爾定理}
% $\mfc{a,b}\mt{連續},\mf{a,b}\mt{可導},f\mf{a}=f\mf{b}\rightarrow \exists \mt{至少一個}\xi \in \mf{a,b},\mt{使得}f^{'}\mf{\xi}=0$

% \mb{拉格朗日中值定理}
% $\mfc{a,b}\mt{連續},\mf{a,b}\mt{可導}\rightarrow \exists \mt{至少一個}\xi \in \mf{a,b},\mt{使得} f\mf{b}-f\mf{a}=f^{'}\mf{\xi}\mf{b-a}$\par
% $\quad \mt{證明構造輔助函數}\varphi\mf{x}$\par
% $\quad \mt{微分中值定理}$\par
% $\quad x \in I \mt{連,導},f^{'}\mf{x}=0 \rightarrow x \in I ,f\mf{x}\equiv C \mf{\mt{常數}}$

% \mb{柯西中值定理}
% $\mfc{a,b}\mt{連續},\mf{a,b}\mt{可導},\forall x \in \mf{a,b},F^{'}\mf{x}\neq 0\rightarrow \exists \mt{至少一點} \xi \in \mf{a,b} \mt{使得} \frac{f\mf{b}-f\mf{a}}{F\mf{b}-F\mf{a}}=\frac{f^{'}\mf{\xi}}{F^{'}\mf{\xi}}$\par
% $\quad \mt{證明構造輔助函數}\varphi\mf{x}$\par

% \subsection{諾必達法則}
% $\left.\ma{
%     when \, x\rightarrow a ,f\mf{x} \, and \, F\mf{x} \rightarrow 0\\
%     when \, x \in U\mf{a}, \exists f^{'}\mf{x} \, and\,  F^{'}\mf{x}, F^{'}\mf{x}\neq 0 \\
%     \exists \md{\lim_{x \rightarrow a}\frac{f^{'}\mf{x}}{F^{'}\mf{x}}},or \, quip \, \infty
% }\right\} \Rightarrow \md{\lim_{x \rightarrow a}\frac{f\mf{x}}{F\mf{x}} =\lim_{x \rightarrow a}\frac{f^{'}\mf{x}}{F^{'}\mf{x}}}$\par
% \quad \par
% \vspace{12 pt}

% $\left.\ma{
%     when \, x\rightarrow \infty ,f\mf{x} \, and \, F\mf{x} \rightarrow 0\\
%     when \, \mfa{x} > N, \exists f^{'}\mf{x} \, and\,  F^{'}\mf{x}, F^{'}\mf{x}\neq 0 \\
%     \exists \md{\lim_{x \rightarrow \infty}\frac{f^{'}\mf{x}}{F^{'}\mf{x}}},or \, quip \, \infty
% }\right\} \Rightarrow \md{\lim_{x \rightarrow \infty}\frac{f\mf{x}}{F\mf{x}} =\lim_{x \rightarrow \infty}\frac{f^{'}\mf{x}}{F^{'}\mf{x}}}$

% \subsection{泰勒公式}

% \noindent
% $\ma{
%     \mt{if } x=x_0 \,\exists f^{\mf{n}}\mf{x_0}\\
%     \mt{then } \exists U\mf{x_0} \,\forall x \in U\mf{x_0} \\
    
% }
% \mt{使得} \ma{ 
%     f\mf{x}&=P_n\mf{x}+R_n\mf{x} \\ 
%            &=f\mf{x_0}+f^{'}\mf{x_0}\mf{x-x_0}+\frac{f^{''}\mf{x_0}}{2!}\mf{x-x_0}^2+\dots+\frac{f^{\mf{n}}\mf{x_0}}{n!}\mf{x-x_0}^n+  R_n\mf{x} \\
%            &\mt{其中} R_n\mf{x}=o\mf{\mf{x-x_0}^n}  \mf{\mt{佩亚诺余项} ,x_0=x \mt{麦克劳林}} 
% }$\par
% \vspace{12 pt}

% \noindent
% $\ma{
%     \mt{if } x=x_0 \,\exists f^{\mf{n+1}}\mf{x_0}\\
%     \mt{then } \exists U\mf{x_0} \,\forall x \in U\mf{x_0} \\
    
% }
% \mt{使得} \ma{ 
%     f\mf{x}&=P_n\mf{x}+R_n\mf{x} \\ 
%            &=f\mf{x_0}+f^{'}\mf{x_0}\mf{x-x_0}+\frac{f^{''}\mf{x_0}}{2!}\mf{x-x_0}^2+\dots+\frac{f^{\mf{n}}\mf{x_0}}{n!}\mf{x-x_0}^n+  R_n\mf{x} \\
%            &\mt{其中} R_n\mf{x}= \frac{f^{\mf{n+1}}\mf{\xi}}{\mf{n+1}!}\mf{x-x_0}^{n+1} ,\xi \in \mf{x_0,x}  \mf{\mt{拉格朗日余项} ,x_0=x \mt{麦克劳林}}
% }$

% % --------------------------------------------------------
% \subsection{函数单调性与曲线凹凸性}

% \subsubsection{函数单调性的判定法}
% $\mfc{a,b}\mt{连续},\mf{a,b}\mt{可导}\ma{
%     \mt{if } x \in \mf{a,b} ,f^{'}\mf{x} \geqslant 0,\mt{有限点取等号 then } y=f\mf{x} \mt{在} \mfc{a,b}\mt{单增}\\
%     \mt{if } x \in \mf{a,b} ,f^{'}\mf{x} \leqslant 0,\mt{有限点取等号 then } y=f\mf{x} \mt{在} \mfc{a,b}\mt{单减}
% }$\par
% $\mt{驻点与无定义点}$

% \subsubsection{曲线的凹凸性与拐点}
% $ f\mf{x} \mt{在}x \in I\mt{上连续},
% \ma{
%     if \q \forall x_1,x_2\in I ,\mt{恒有}f\mf{\frac{x_1+x_2}{2}}<\frac{f\mf{x_1}+f\mf{x_2}}{2}\mt{ then } f\mf{x}\mt{在}I \mt{内图像凹}\\
%     if \q \forall x_1,x_2\in I ,\mt{恒有}f\mf{\frac{x_1+x_2}{2}}>\frac{f\mf{x_1}+f\mf{x_2}}{2}\mt{ then } f\mf{x}\mt{在}I \mt{内图像凸}
%     }
% $\par



% \vspace{12 pt}

% $f\mf{x}\mt{在}\mfc{a,b}\mt{连续},\mt{在}\mf{a,b}\mt{一阶导,二阶导}\ma{
%     if \q x\in \mf{a,b} \q f^{''}\mf{x}>0, then \q f\mf{x} \mt{在} \mfc{a,b} \mt{内图像凹} \\
%     if \q x\in \mf{a,b} \q f^{''}\mf{x}<0, then \q f\mf{x} \mt{在} \mfc{a,b} \mt{内图像凸} 
% }$\par

% $f^{''}\mf{x}=0 \mt{点 and } f^{''}\mf{x}\mt{不存在点,点两侧异号为拐点}$





% \subsection{函数的极值与最大值最小值}
% \mb{函数的极值及其求法}\par
% $f\mf{x} \q x \in \mathring{U}\mf{x_0},if \q \forall x \in \mathring{U}\mf{x_0} \q f\mf{x}<f\mf{x_0}, then \q f\mf{x_0} \mt{是函数}f\mf{x}\mt{的一个极大值}$\par
% $f\mf{x} \q x \in \mathring{U}\mf{x_0},if \q \forall x \in \mathring{U}\mf{x_0} \q f\mf{x}>f\mf{x_0}, then \q f\mf{x_0} \mt{是函数}f\mf{x}\mt{的一个极小值}$\par
% \qa
% $if \q f\mf{x}\mt{在} x_0 \mt{处可导,取得极值} ,then \q f^{'}\mf{x}=0$\par

% \qa
% $\ma{
%     f\mf{x}\mt{在} x_0 \mt{处连续}\\
%     \mt{在} \mathring{U}\mf{x_0,\delta} \mt{内可导}
% }
%  \ma{
%     if \q x\in \mf{x_0-\delta,x_0},f^{'}\mf{x}>0 \q x\in \mf{x_0 ,x_0+\delta},f^{'}\mf{x}<0, then \q f\mf{x} \mt{在}x_0\mt{处取得极大值}\\
%     if \q x\in \mf{x_0-\delta,x_0},f^{'}\mf{x}<0 \q x\in \mf{x_0 ,x_0+\delta},f^{'}\mf{x}>0, then \q f\mf{x} \mt{在}x_0\mt{处取得极小值}\\
%     if \q x\in \mathring{U}\mf{x_0,\delta},f^{'}\mf{x}\mt{的符号保持不变},then \q f\mf{x} \mt{在}x_0\mt{处没有极值}
% }$\par
% \qa

% $f^{'}\mf{x}=0 \mt{点 and } f^{'}\mf{x}\mt{不存在点,点两侧异号为极值点}$\par
% \qa

% $f\mf{x} \mt{在}x_0\mt{处具有二阶导数},f^{'}\mf{x_0}=0,f^{''}\mf{x_0}\neq 0\ma{
%     if \q f^{''}\mf{x}<0,then \q f\mf{x}\mt{在}x_0\mt{处取得极大值}\\
%     if \q f^{''}\mf{x}<0,then \q f\mf{x}\mt{在}x_0\mt{处取得极大值}\\
% }$\par

% \mb{最大值最小值问题}\par
% $\mt{驻点值} a_{0,1,2,\dots,n} , \q \mt{不可导点值} b_{0,1,2,\dots,n}, \q \mt{端点值} f\mf{a}, f\mf{b} \rightarrow \ma{ MAX=max\mfb{a,b,f\mf{a}, f\mf{b}}\\ MIN=min\mfb{a,b,f\mf{a}, f\mf{b}}}$\par

% \subsection{函数图像的绘制}
% $f^{'}\mf{x}=0,f^{''}\mf{x}=0\mt{点},\mt{水平,铅锤}$

% \subsection{曲率}
% \mb{弧微分}\par
% $\frac{\Delta s}{\Delta x}=\sqrt{\mf{\frac{\mh{MM^{'}}{\frown}}{\mfa{MM^{'}}}}\cdot\mfc{1+\mf{\frac{\Delta y}{\Delta x}}^2}}$\par
% \qa
% $\ma{
%     \frac{ds}{dx} &=\md{\lim_{\Delta x \rightarrow x}\frac{\Delta s}{\Delta x}}\\
%     &=\sqrt{1\cdot\mfc{1+{y^{'}}^2}}
% }$\par

% \mb{曲率及计算公式}\par
% $\overline{K}=\mfa{\frac{\Delta \alpha}{ \Delta s}}$\par
% $\ma{
%     K &=\md{\lim_{\Delta s \rightarrow 0}\mfa{\frac{\Delta \alpha}{ \Delta s}}}\\
%         &=\md{\frac{\mfa{\frac{d \alpha}{dx}}}{\frac{ds}{dx}}= \frac{\mfa{y^{''}}}{1+{y^{'}}^2} / \sqrt{1+{y^{'}}^2} }
%         =\ma{
%             \mt{直线}  \q &0\\
%             \mt{圆}   \q &\frac{1}{a}\\
%             \mt{参数方程} \q &\frac{\mfa{g^{''}f^{'}-g^{'}f^{''}}}{\mf{{g^{'}}^2+{f^{'}}^2}^{\frac{3}{2}}}
%         }
%     }
% $

% \mb{曲率圆与曲率半径}\par
% $\rho=\frac{1}{K}$\par

% \mb{曲率中心的计算公式 $\q$ 渐屈线与渐伸线}\par
% $\mt{曲率中心}D\mf{\alpha,\beta}\left\{ \ma{
%     \alpha &=x-\frac{y^{'}\mf{1+{y^{'}}^2}}{y^{''}}\\
%     \beta  &=y+\frac{1+{y^{'}}^2}{y^{''}}\\
% }\right.$



% \subsection{方程近似解}

% \mb{二分法}\par
% $\mfc{a,b}\mt{连续},f\mf{a}\cdot f\mf{b}<0,\exists \mt{一个实根} \xi \in \mf{a,b},\xi_n=\frac{a+b}{2}$\par
% \mb{切线法}\par
% $f\mf{a}\cdot f\mf{b}<0 ,x \in \mfc{a,b}  f^{'}\mf{x} f^{''}\mf{x}\mt{保持定号}\rightarrow \exists \mt{实根}\xi \in \mfc{a,b}$\par
% $x_0=\mt{a or b},f\mf{x_0}\cdot f^{''}\mf{x}>0,x_n \mt{为过} x_{n-1}\mt{的切线与} x\mt{轴交点},x_{n+1}=x_n-\frac{f\mf{x_n}}{f^{'}\mf{x_n}}$

% \mb{割线法}\par
% $\mt{初始值} x_0, x_1, x_{n+1}=x_n-\frac{x_n-x_{n-1}}{f\mf{x_n}-f\mf{x_{n-1}} }\cdot f\mf{x_n}$



% \newpage
% \section{不定积分}
% \subsection{不定积分的概念与性质}
% \mb{原函数与不定积分的概念}\par
% $if \q \forall x \in I ,F^{'}\mf{x}=f\mf{x},dF\mf{x}=f\mf{x}dx ,then \q F\mf{x}\mt{为}f\mf{x}\mt{在区间}I\mt{原函数}$\par
% $if \q x \in I ,f\mf{x}\mt{连续}, then \q \exists \mt{可导函数}F\mf{x},\forall x \in I ,F^{'}\mf{x}=f\mf{x} \mt{连续函数一定有原函数}$\par


% \qa

% $\int f\mf{x}dx=F\mf{x}+C\mt{不定积分} \q \left\{\ma{
%     \int  \q \mt{积分符号}\\
%     f\mf{x}  \q  \mt{被积函数}\\
%     f\mf{x}dx  \q  \mt{被积表达式}\\
%     x \q  \mt{积分变量}
% }\right.$\par

% \mb{基本积分表}\par

% \dots

% \mb{不定积分的性质}\par
% $\int \mfc{f\mf{x}+g\mf{x}} dx = \int f\mf{x} dx + \int g\mf{x} dx$\par
% $\int k f\mf{x} dx=k\int f\mf{x} dx$\par


% \subsection{积分换元法}

% \mb{第一类换元法}\par
% $if \q F^{'}\mf{\mu}=f\mf{\mu},\mu=\varphi \mf{x}\mt{可导},then \q \int f\mfc{\varphi\mf{x}}\varphi^{'}\mf{x}dx=\mfc{\int f\mf{\mu}d\mu}_{u=\varphi\mf{x}}$

% \mb{第二类换元法}\par
% $if  \q x=\varphi\mf{t}\mt{单调可导},\varphi\mf{t}\neq 0,f\mfc{\varphi\mf{t}}\varphi^{'}\mf{t}\mt{具有原函数},then \q \int f\mf{x}dx=\mfc{\int f\mf{\varphi\mf{t}}\varphi^{'}\mf{t}dt}_{t=\varphi^{-1}\mf{x}}$

% \subsection{分部积分法}
% $\int uv^{'}dx=uv-\int u^{'}vdx \rightarrow \int udv=uv-\int uvdu$

% \subsection{有理函数的积分}
% \mb{有理函数的积分}\par
% $\ma{
%      \mt{两多项式的商}\frac{P\mf{x}}{Q\mf{x}}\mt{称为有理函数,有理分式}\\
%       P\mf{x} \mt{次数小于}Q\mf{x}\mt{真分式},\mt{否则假分式}\\
%       \mt{假分式}= \mt{多项式}+ \mt{真分式}
% }$

% \mb{可化为有理函数的积分举例}\par
% $\sin x=\frac{2\tan \frac{x}{2}}{1+\tan^2\mf{\frac{x}{2}}}$\par
% $\cos x=\frac{1-\tan^2\mf{\frac{x}{2}}}{1+\tan^2\mf{\frac{x}{2}}}$\par

% \subsection{积分表的使用}
% \dots

% \newpage
% \section{定积分}
% \subsection{定积分的概念与性质}
% \subsubsection{定积分的概念与性质}
% \mb{曲边梯形的面积}\p
% $\ma{
%     A &\approx\md{\sum_{i=1}^n f\mf{\xi_i}\Delta x_i}\\
%     &= \md{\lim_{\lambda \rightarrow 0}\sum_{i=1}^n f\mf{\xi_i}\Delta x_i}\mf{\lambda=max\mfb{\Delta x_1, \Delta x_2,\cdots, \Delta x_n },\Delta x_n=x_n-x_{n-1}}
% }$\p


% \mb{变速直线运动的路程}\p
% $\ma{
%     A &\approx\md{\sum_{i=1}^n v\mf{\tau_i} \Delta  t_i}\\
%     &= \md{\lim_{\lambda \rightarrow 0}\sum_{i=1}^n v\mf{\tau_i}\Delta t_i}\mf{\lambda=max\mfb{\Delta t_1, \Delta t_2,\cdots, \Delta t_n },\Delta t_n=t_n-t_{n-1}}
% }$\p


% \subsubsection{定积分的定义}
% $\md{\int_a^bf\mfb{x}dx=I=\lim_{\lambda \rightarrow 0}\sum_{i=1}^nf\mf{\xi_i}\Delta x_i}\left\{ 
%   \ma{  
%         f\mfc{x} \mt{被积函数}\\
%         f\mfc{x}dx \mt{被积表达式}\\
%         x \mt{积分变量}\\
%         a \mt{积分下限}\\ 
%         b \mt{积分上限} \\
%         \mfc{a,b} \mt{积分区间} 
%   }
% \right.$\p

% $\mt{和式(积分和)}\md{\lim_{\lambda \rightarrow 0}\sum_{i=1}^nf\mf{\xi_i}\Delta x_i}\mt{极限存在时,极限}I\mt{只与}f\mfc{x},\mfc{a,b}\mt{有关}$\p
% $\mt{积分和存在,函数}f\mf{x}\mt{可积}$\p
% $\mfc{a,b}\mt{连} , \mfc{a,b}\mt{可积}$\p
% $\mfc{a,b}\mt{有界,有限个间断点} , \mfc{a,b}\mt{可积}$\p

% \subsubsection{定积分的近似计算}

% \subsubsection{定积分的性质}
% $b=a,\md{\int_a^bf\mf{x}dx=0};\q a>b,\md{\int_a^bf\mf{x}dx=-\int_b^af\mf{x}dx}$\p
% $a,b\mt{常数},\md{\int_a^b\mfc{af\mf{x}+bg\mf{x}}dx=a\int_a^b f\mf{x}dx + b\int_a^b g\mf{x}dx}$\p
% $a<c<b,\int_a^bf\mf{x}dx=\int_a^cf\mf{x}dx + \int_c^bf\mf{x}dx\mt{积分区间可加}$
% $\int_a^b1dx=\int_a^bdx=b-a$\p
% $if \q x \in \mfc{a,b},f\mf{x}\geqslant 0,then \q \int_a^b f\mf{x}dx\geqslant 0$\p
% $if \q x \in \mfc{a,b},f\mf{x}\leqslant g\mf{x},then \q \int_a^b f\mf{x}dx \leqslant \int_a^b g\mf{x}dx $\p
% $\mfa{\int_a^b f\mf{x}dx }\leqslant \int_a^b \mfa{f\mf{x}}dx $\p
% $m<f\mf{x} <M,m\mf{b-a}<\int_a^b f\mf{x}dx<M\mf{b-a}$\p
% $if \q f\mf{x}\mt{在} \mfc{a,b}\mt{连续},then \q \exists \mt{至少一个点} \xi \in \mfc{a,b} ,\int_a^bf\mf{x}dx=f\mf{\xi}\mf{b-a}$\p

% \subsection{微积分基本公式}
% \subsubsection{变速直线运动中位置函数与速度函数}
% $F^{'}\mf{x}=f\mf{x},\int_a^bf\mf{x}dx=F\mf{b}-F\mf{a}$

% \subsubsection{积分上限的函数及其导数}

% $if \q f\mf{x}在\mfc{a,b}连续,then \q \Phi \mf{x}=\int_a^xf\mf{t}dt\mt{在}\mfc{a,b}\mt{可导},\Phi^{'}\mf{x}=\frac{d}{dx}\cdot \int_a^xf\mf{t}dt=f\mf{x}$\p
% $if \q f\mf{x}在\mfc{a,b}连续,then \q  \Phi \mf{x}=\int_a^xf\mf{t}dt\mt{是}f\mf{x}\mt{在}\mfc{a,b}\mt{上的原函数}$\p

% \subsubsection{牛顿-莱布尼兹}
% $if \q x \in \mfc{a,b},F^{'}\mf{x}=f\mf{x},\int_a^bf\mf{x}dx=F\mf{b}-F\mf{a}\mt{微积分基本公式}$\p

% \subsection{定积分的换元法和分部积分法}

% \subsubsection{定积分的换元法}

% $f\mf{x}\mt{在}\mfc{a,b}\mt{连续},if  \q 
%     \ma{
%         \varphi\mf{\alpha}=a,\varphi\mf{\beta}=b,\\
%         \varphi\mf{x}\mt{在}\mfc{a,b}\mt{具有连续导数},\\
%         R_{\varphi}=\mfc{a,b}
%     } 
% then \q \int_a^bf\mf{x}dx=\int_\alpha^\beta f\mfc{\varphi\mf{t}}\varphi^{'}\mf{t}dt$

% \qa
% $\int_a^b f\mfc{\varphi\mf{x}}\varphi^{'}\mf{x}dx=\int_\alpha^\beta f\mf{t}dt$

% \subsubsection{定积分的分布积分法}

% $\int_a^b uv^{'}dx=\mfc{a,b}_a^b-\int_a^bu^{'}vdx \q \int_a^b udv=\mfc{a,b}_a^b-\int_a^bvdu$



% \subsection{反常积分}
% \subsubsection{无穷限的反常积分}

% \noindent
% $\ma{\mt{无穷限的反常积分}\\
%     \mf{\mt{极限存在则收敛}}\\
%     F^{'}\mf{x}=f\mf{x}
% }
% \left\{\ma{
%     \int_a^{+\infty}f\mf{x}dx=\md{\lim_{t \rightarrow \infty}}\int_a^tf\mf{x}dx=\lim_{t \rightarrow +\infty}F\mf{t}-F\mf{a}=I_1\\
%     \int_{-\infty}^bf\mf{x}dx=\md{\lim_{t \rightarrow -\infty}}\int_t^bf\mf{x}dx=F\mf{b}-\lim_{t \rightarrow -\infty}F\mf{t}=I_2\\
%     \int_{-\infty}^{+\infty}f\mf{x}dx=\md{\lim_{t \rightarrow -\infty}\int_{-\infty}^0f\mf{x}dx + \lim_{t \rightarrow +\infty}\int_0^{+\infty}f\mf{x}dx=\lim_{t \rightarrow +\infty}F\mf{t} - \lim_{t \rightarrow -\infty}F\mf{t}=I_2+I_2}
% }\right.$

% \subsubsection{无界函数的反常积分}

% \noindent
% $if \q  \ma{
%     f\mf{x}\mt{在a的任一领域无界}\\
%     x \in U\mf{a,\delta},\forall \delta>0,M>0, f\mf{x}>M}
% ,then \q a\mt{称为} f\mf{x} \mt{瑕点}, f\mf{x}\mt{积分称为瑕积分}$\p
% $\ma{\mt{无界函数的反常积分}\\
% \mf{\mt{极限存在则收敛}}\\
% F^{'}\mf{x}=f\mf{x}
% }
% \left\{\ma{
%     \md{\int_a^bf\mf{x}dx=\lim_{t \rightarrow b^-}\int_a^tf\mf{x}dx}=\lim_{t \rightarrow b^-}F\mf{t}-F\mf{a} \q x \in \left [ a,b\right) \\
%     \md{\int_a^bf\mf{x}dx=\lim_{t \rightarrow a^+}\int_t^bf\mf{x}dx}=F\mf{b}-\lim_{t \rightarrow a^+}F\mf{t}  \q x \in \left(a,b \right ] \\ 
%     \md{\int_a^bf\mf{x}dx=\lim_{t \rightarrow a^+}\int_t^cf\mf{x}dx+\lim_{t \rightarrow b^-}\int_c^tf\mf{x}dx}=\lim_{t \rightarrow b^-}F\mf{t} - \lim_{t \rightarrow a^+}F\mf{t} \q x \in \mf{a,b}
% }\right.$

% \subsection{反常积分的审敛法 $\q \Gamma $函数}

% \subsubsection{无穷限反常积分的审敛法}
% $f\mf{x} \mt{在} [a,+\infty) \mt{连续},f\mf{x} \geqslant 0,if \q F\mf{x}=\int_a^xf\mf{t}dt \mt{在} [a,+\infty)\mt{有上界},then \q \int_a^{+\infty}f\mf{x}dx\mt{收敛}$\p
% \qa

% $f\mf{x},g\mf{x} \mt{在} [a,+\infty) \mt{连续},0\leqslant f\mf{x} \leqslant g\mf{x}\ma{
%     if \q \int_a^{+\infty}g\mf{x}dx\mt{收敛},then \q \int_a^{+\infty}f\mf{x}dx\mt{收敛}\\
%     if \q \int_a^{+\infty}f\mf{x}dx\mt{发散},then \q \int_a^{+\infty}g\mf{x}dx\mt{发散}
% }$\p
% \qa

% $f\mf{x} \mt{在} [a,+\infty)\mf{a>0} \mt{连续},f\mf{x}\geqslant 0 \ma{
%     if \q \exists M>0,p>1,f\mf{x}\leqslant \frac{M}{x^p}\mf{a\leqslant x<+\infty},then \q \int_a^{+\infty}f\mf{x}dx\mt{收敛}\\
%     if \q \exists N>0,p\leqslant 1,f\mf{x}\geqslant \frac{N}{x^p}\mf{a\leqslant x<+\infty},then \q \int_a^{+\infty}f\mf{x}dx\mt{发散}
% }$\p
% \qa

% $f\mf{x} \mt{在} [a,+\infty) \mt{连续},f\mf{x}\geqslant 0\ma{
%     if \q \exists p>1,\md{\lim_{x \rightarrow +\infty}x^pf\mf{x}=c<+\infty,then \q \int_a^{+\infty}f\mf{x}dx\mt{收敛}}\\
%     if \q \exists p\leqslant 1,\md{\lim_{x \rightarrow +\infty}x^pf\mf{x}=d>0\mf{or +\infty},then \q \int_a^{+\infty}f\mf{x}dx\mt{发散}}
% }$\p
% \qa
% $f\mf{x} \mt{在} [a,+\infty) \mt{连续},if \q \int_a^{+\infty}\mfa{f\mf{x}}dx\mt{收敛},then \q \int_a^{+\infty}f\mf{x}dx\mt{收敛,绝对收敛}$

% \subsubsection{无界函数反常积分的审敛法}

% \noindent
% $f\mf{x} \mt{在} (a,b] \mt{连续},f\mf{x}\geqslant 0 ,x=a\mt{为函数瑕点}\ma{
%     if \q \exists M>0,p<1,f\mf{x}\leqslant \frac{M}{\mf{x-a}^p}\mf{a\leqslant x<+\infty},then \q \int_a^{b}f\mf{x}dx\mt{收敛}\\
%     if \q \exists N>0,p\geqslant 1,f\mf{x}\geqslant \frac{N}{\mf{x-a}^p}\mf{a\leqslant x<+\infty},then \q \int_a^{b}f\mf{x}dx\mt{发散}
% }$\p

% \qa
% \noindent
% $f\mf{x} \mt{在} (a,b] \mt{连续},f\mf{x}\geqslant 0,x=a\mt{为函数瑕点}\ma{
%     if \q \exists 0<p<1,\md{\lim_{x \rightarrow a^+}\mf{x-a}^pf\mf{x}=c<+\infty,then \q \int_a^bf\mf{x}dx\mt{收敛}}\\
%     if \q \exists p\geqslant 1,\md{\lim_{x \rightarrow a^+}\mf{x-a}^pf\mf{x}=d>0\mf{or +\infty},then \q \int_a^bf\mf{x}dx\mt{发散}}
% }$\p

% \subsubsection{$\Gamma 函数$}

% $\ma{
%     \Gamma\mf{s}=\int_0^\infty e^{-x}x^{s-1}dx\\
%     \Gamma\mf{s+1}=s\Gamma\mf{s}\mf{\mt{分部积分法}}\\
%     \Gamma\mf{1}=1,\Gamma\mf{n+1}=n!\\
%     \md{\lim_{s\rightarrow 0^+}}\Gamma\mf{s}=\md{\lim_{s\rightarrow 0^+}}\frac{\Gamma\mf{1+s}}{s} =+\infty\\
%     \Gamma\mf{s}\Gamma\mf{1-s}=\frac{\pi}{\sin \pi s}\mf{0<s<1,\mt{余元公式}},\Gamma\mf{\frac{1}{2}}=\sqrt{\pi}\\
%     \Gamma\mf{\frac{1}{2}}=\int_0^\infty e^{-x}x^{-\frac{1}{2}}dx \xlongequal{x=u^2}
%     \int_0^\infty e^{-u^2}\mf{u^2}^{-\frac{1}{2}}d\mf{u^2}=2\int_0^\infty e^{-u^2}du
% }$\p
 
% \newpage
% \section{定积分的应用}
% \subsection{定积分的元素法}
% $\ma{
%     A \approx \sum f\mf{x}dx\\
%     =\lim \sum f\mf{x}dx=\int_a^b f\mf{x}dx
% } $

% \subsection{定积分在几何学上的应用}
% \subsubsection{平面图形}
% \mb{直角坐标}\par
% $\mt{矩形面积} A=xy$\p
% $A=\int_a^bf\mf{x}dx=\int_c^df\mf{y}dy$\p

% \mb{极坐标}\par
% $\mt{扇形面积} A=\frac{1}{2}R^2\theta$\p
% $A=\int_a^b\frac{1}{2}\mfc{\rho\mf{\theta}}^2d\theta$

% \subsubsection{体积}

% \mb{旋转体体积}\par
% $V=\int_a^b \pi \mfc{f\mf{x}}^2dx$

% \mb{平行截面积已知的立体体积}\par
% $V=\int_a^b A\mf{x}dx=\int_c^d A\mf{y}dy$

% \subsubsection{平面曲线的弧长}


% $if \q \exists \md{\sum_{i=1}^n\mfa{M_{i-1}^i}}\mt{极限存在},then \q \mt{极限称为}\mh{AB}{\frown}\mt{的弧长}$\p
% 光滑曲线弧是可求长的\p
% $\ma{
%     ds &=\sqrt{\mf{dx}^2+\mf{dy}^2}\\
%         &=\sqrt{\mf{\varphi^{'}\mf{t} dt}^2+\mf{ \psi ^{'}\mf{t} dt}^2}=\sqrt{{\varphi^{'}\mf{t}}^2+{\psi ^{'}\mf{t}}^2}dt\\
%         &=\sqrt{\mfc{\mf{\rho^{'}\cos \theta-\rho \sin \theta} dt}^2+\mfc{\mf{\rho^{'}\sin \theta+\rho \cos \theta }dt}^2}=\sqrt{\mf{\rho}^2+\mf{\rho^{'}}^2}dt
% }$


% \subsection{定积分在物理学上的应用}

% \subsubsection{变力沿直线做功}
% $\ma{W &=F\cdot s\\
%        &=\frac{kq^2}{r^2}dr
% }$

% \subsubsection{水压力}

% $\ma{
%     p=\rho gh, P &=p\cdot A\\
%       &=\rho g x\cdot 2\sqrt{R^2-x^2}dx
% }$

% \subsubsection{引力}
% $\ma{
%     F &=G\frac{m_1m_2}{r^2}\\
%       &=G\frac{m_1\mu dy}{a^2+y^2}
% }
% $

\setcounter{section}{6}

\newpage


\section{微分方程}
\subsection{基本概念}
$
\left\{
    \begin{array}{ll}
        \frac{dy}{dx}=2x\\
        x=1,y=2
    \end{array}
\right.\Rightarrow  y=2x+1
$\par
函數,函數導數,自變量關係的方程\textbf{微分方程}\par
最高階導數的階數\textbf{微分方程的階}\par
$F\left(x,y^{'},\dots,y^{\left( n \right)  } \right)=0  $\textbf{一般形式}\par
函數\textbf{微分方程的解}\par
函數含常數,常數個數同階數\textbf{微分方程的通解}\par
確定常數的通解\textbf{微分方程的通解}
\subsection{可分離變量的微分方程}
$\frac{dy}{dx}=2x\Rightarrow dy=2x dx \Rightarrow $\par
$\frac{dy}{dx}=2xy^2\Rightarrow \frac{dy}{y^2}=2xdx\Rightarrow$\par
$g\left( y \right)dy=f\left( x \right)dx \Rightarrow y=\varphi \left( x \right)$\par
$G\left( y \right)=F\left( x \right)+C \Rightarrow y=\varPhi  \left( x \right)$
\subsection{齊次方程}
$\frac{dy}{dx}=\varphi\left(\frac{y}{x} \right) $\par
$\frac{dy}{dx}=\frac{ax+by+c}{a_1x+b_1y+c_1}$

\subsection{一階線性微分方程}
$\frac{dy}{dx}+P\left(x \right)y=Q\left(x \right)$關於函數,函數導數是一次方程\textbf{一階線性微分方程}\par
$Q\left(x \right)=0$時\textbf{齊次方程}\par
$y=Ce^{-\int P\left(x \right)dx }$\textbf{齊次通解},設$C=\mu \left(x\right)$\par
$\mu \left(x\right)=\int Q\left(x\right)e^{\int P\left(x \right)\,dx }\,dx+C_2 $\par
$y=\left(\int Q\left(x\right)e^{\int P\left(x \right)\,dx }\,dx+C_2\right)e^{-\int P\left(x \right)dx } $\textbf{非齊次通解}\par
$\frac{dy}{dx}+P\left(x\right)y=Q\left(x\right)y^n$\textbf{伯努利方程}\par
$\frac{dy}{dx}y^{-n}+P\left(x\right)yy^{-n}=Q\left(x\right)\,
\underrightarrow{z=y^{1-n},z^{'}=\left(1-n\right)y^{-n}\frac{dy}{dx}}
\frac{z^{'}}{\left(1-n\right)}+P\left(x\right)z=Q\left(x\right)
\textbf{一階線性微分方程}
$
\subsection{可降階的高階微分方程}
$y^{\left(n\right)}=f\left(x\right) \quad
\Rightarrow \quad
y^{\left(n-1\right)}=\int f\left(x\right)\,dx 
$\par

$y^{''}=f\left(x,y^{'}\right)\quad
\underrightarrow{y^{'}=p , y^{''}=p^{'}}\quad
p^{'}=f\left(x,p\right) 
$\par

$y^{''}=f\left(y,y^{'}\right)\quad
\underrightarrow{y^{'}=p , y^{''}=p\frac{dp}{dy}}\quad
p\frac{dp}{dy}=f\left(y,p\right) \quad
\underrightarrow{p=y^{,}=\varphi\left(y,c_1\right)}\quad
$

\subsection{高階性微分方程}
二阶微分方程\par
$$\frac{d^2y}{dx^2}+P\left(x\right)\frac{dy}{dx}+Q\left(x\right)y=f\left(x\right)$$\par
解的结构\par
$y=C_1y_1\left(x\right)+C_2y_2\left(x\right)$是解\par
$y=C_1y_1\left(x\right)+C_2y_2\left(x\right)$无关特解是通解\par
$y=C_1y_1\left(x\right)+C_2y_2\left(x\right)+\dots + C_ny_n\left(x\right) $无关特解是通解\par
$y=Y\left(x\right)+y^*\left(x\right)$齐次通,非齐特,非齐通\par
$$\frac{d^2y}{dx^2}+P\left(x\right)\frac{dy}{dx}+Q\left(x\right)y=f_1\left(x\right)+f_2\left(x\right)$$\par
$y=y_1^*\left(x\right)+y_2^*\left(x\right)$特解,特解,特解\par
常数变易法\par
$Y\left(x\right)=C_1y_1\left(x\right)+C_2y_2\left(x\right)$\par
$Y\left(x\right)=v_1\left(x\right)y_1\left(x\right)+v_2\left(x\right)y_2\left(x\right)$\par
$
\left\{
    \begin{array}{ll}
        y_1v_1^{'}+y_2v_2^{'}=0\\
        y_1^{'}v_1^{'}+y_2^{'}v_2^{'}=f
    \end{array}
\right.\Rightarrow 
\begin{vmatrix}
    y_1 & y_2 &0\\
    y_1^{'} & y_2^{'} &f
\end{vmatrix}
\underrightarrow{W=y_1y_2^{'}-y_1^{'}y_2}
\left\{ 
\begin{array}{ll}
    v_1^{'}=-\frac{y_2f}{W}\\
    v_2^{'}=\frac{y_1f}{W}  
\end{array}
\right.\Rightarrow
\left\{ 
\begin{array}{ll}
    v_1=-\int \frac{y_2f}{W}\,dx+c_1\\
    v_2=\int \frac{y_1f}{W}\,dx+c_2  
\end{array}
\right.
$\par

$\left\{ 
    \begin{array}{ll}
        v_1y_1=\left(-\int \frac{y_2f}{W}\,dx+c_1\right)y_1\\
        v_2y_2=\left( \int \frac{y_1f}{W}\,dx+c_2 \right)y_2  
    \end{array}\Rightarrow
    Y\left(x\right)=c_1y_1+c_2y_2-y_1\int \frac{y_2f}{W}\,dx
    +y_2\int \frac{y_1f}{W}\,dx
\right.$

\subsection{常系数齐次线性微分方程}
$y^{''}+P\left(x\right)y^{'}+Q\left(x\right)y=0\Rightarrow
y^{''}+py^{'}+qy=0$\par
$y=e^{rx}$\par
$(e^{rx})^{'}=re^{rx}$\par
$\left(r^2+pr+q\right)e^{rx}=0$\par

$\left\{
    \begin{array}{ll}
        p^2-4q>0,r_{1,2}=\frac{-p \pm \sqrt{p^2-4q} }{2}\\
        p^2-4q=0,r_1=r_2=\frac{-p }{2},
            y_2=e^{r_1x}\mu\left(x\right),\mu^{''}=0\\
        p^2-4q<0,r_{1,2}=\alpha\pm \beta i,
            e^{(i\theta)}=\cos\left(\theta\right)+i\sin \left(\theta\right),
            \overline{y_1} =\frac{1}{2}\left(y_1+y_2\right),
            \overline{y_1} =\frac{1}{2i}\left(y_1-y_2\right)        
    \end{array}
\right.
$\par
$\left\{
    \begin{array}{ll}
        p^2-4q>0, y=C_1y_1+C_2y_2=C_1e^{r_1x}+C_2e^{r_2x}\\
        p^2-4q=0, y=C_1y_1+C_2\mu y_1=\left(C_1x+C_2\right)e^{r_1x}\\
        p^2-4q<0, y=C_1\overline{y_1}+C_2\overline{y_2}=e^{\alpha x}\left(C_1\cos\left(\beta x \right) + C_2\sin\left(\beta x \right) \right)         
    \end{array}
\right.
$\par
n阶常系数齐次微分方程\par
$y^{n}+p_1y^{n-1}+p_2y^{n-2}+\dots+p_{n-1}y^{'}+p_ny=0$\par
$\left\{
    \begin{array}{ll}
        D,\frac{d}{dx}\\
        Dy,\frac{dy}{dx}\\
        D^ny,\frac{d^ny}{dx^n}
    \end{array}
\right.$\par

$\left(D^n+p_1D^{n-1}+\dots+p_{n-1}D+p_n \right)y=0\Rightarrow
L\left(D\right)y=0,
$微分算子D的n次多项式\par
$De^{rx}=re^{rx},\dots,D^ne^{rx}=r^ne^{rx}\Rightarrow
\left(r^n+p_1r^{n-1}+\dots+p_{n-1}r+p_n \right)e^{rx}=L(r)e^{rx}=0\Rightarrow L(r)=0
$\par

$\left\{ 
    \begin{array}{ll}
        \text{单实根}   &\quad Ce^{rx}\\
        \text{k重实根}  &\quad \left( C_1+C_2x+\dots +C_kx^{k-1}\right)e^{rx}\\
        \text{单复根}   &\quad e^{\alpha x}\left(C_1\cos\left(\beta x \right) +D_1\sin\left(\beta x \right)\right)\\
        \text{k重复根}  &\quad e^{\alpha x}\left(\left(C_1+C_2x+\dots+C_kx^{k-1} \right)\cos\left(\beta x \right) +\left(D_1+D_2x+\dots+D_kx^{k-1} \right)\sin\left(\beta x \right)\right)\\
    \end{array}
\right.$

\subsection{常系数非齐次线性微分方程}
$y^{''}+py^{'}+qy=f\left(x\right)$\par

$p_m\left(x\right)=a_0x^m+a_1x^{m-1}+\dots+a_m$\par

$
\left\{
\begin{array}{l}
e^{\theta i}=\cos\left(\theta\right)+i\sin\left(\theta\right)\\
e^{-\theta i}=\cos\left(\theta\right)-i\sin\left(\theta\right)
\end{array}
\right.
\Rightarrow
\left\{
\begin{array}{l}
    \cos\left(\theta\right)=\frac{1}{2}\left(e^{\theta i}+e^{-\theta i}\right)\\
    \sin\left(\theta\right)=\frac{1}{2i}\left(e^{\theta i}-e^{-\theta i}\right)
\end{array}
\right.
\Rightarrow
\left\{
\begin{array}{l}
    \cos\left(\omega x \right)P=\frac{P}{2}\left(e^{\omega x i}+e^{-\omega x i}\right)\\
    \sin\left(\omega x \right)Q=\frac{Q}{2i}\left(e^{\omega x i}-e^{-\omega x i}\right)
\end{array}
\right.
$\par
函数(多项式)共轭,倒数共轭;\quad
两对共轭,乘积共轭;\quad
$e^{\alpha+ \theta i}\text{与} e^{\alpha -\theta i}$共轭\par


\begin{tabular}{|l|}
    \hline
    $f\left(x\right)=e^{\lambda x}P_m\left(x\right)\quad
    \begin{array}{l}
    y^*=R_?\left(x\right)e^{\lambda x}\quad
    \\R^{''}x+\left(2\lambda+p \right)R^{'}x+\left(\lambda^2+p\lambda +q\right)R\left(x\right)=p_m\left(x\right)\quad
    \\y^*=x^kP_m\left(x\right)e^{\lambda x}
    \end{array}
    $\\
    \hline
    $f\left(x\right)=e^{\lambda x}\left[P_l\left(x\right) \cos\left(\omega x  \right) +Q_n\left(x\right)  \sin\left(\omega x  \right) \right]\quad
    \begin{array}{l}
    f\left(x\right)=e^{\lambda x}\left[
        \left(\frac{P}{2}+\frac{Q}{2i}\right)e^{\omega xi} +
        \left(\frac{P}{2}-\frac{Q}{2i}\right)e^{-\omega xi} 
        \right]
    \\f\left(x\right)=\left(\frac{P}{2}+\frac{Q}{2i}\right)e^{\lambda x+ \omega xi} +
    \left(\frac{P}{2}-\frac{Q}{2i}\right)e^{\lambda x-\omega xi} 
    \\f\left(x\right)=\left(\frac{P}{2}+\frac{Q}{2i}\right)e^{(\lambda + \omega i)x} +
    \left(\frac{P}{2}-\frac{Q}{2i}\right)e^{(\lambda -\omega i)x} 
    \\f\left(x\right)=P_1e^{(\lambda + \omega i)x} +
    Q_2e^{(\lambda -\omega i)x} \left(P_1,Q_2\text{共轭}\right)
    \\y_1^*=x^kR_me^{(\lambda+\omega i)x}\left(m =max\left\{P_l,Q_n\right\}\right)
    \\y_2^*=x^k\overline{R_m}e^{(\lambda-\omega i)x}
    \\y^*=y_1^*+y_2^*=x^ke^{\lambda x}\left(R_me^{\omega xi}+\overline{R_m}e^{-\omega xi}\right)
    \\y^*=x^ke^{\lambda x}\left[R_m\left( 
      \cos\left(\omega x\right)+i\sin\left(\omega x\right)  
    \right) +\overline{R_m}\left(
      \cos\left(\omega x\right)-i\sin\left(\omega x\right)  
    \right)\right]
    \\y^*=x^ke^{\lambda x}\left(R_m^{(1)} 
      \cos\left(\omega x   
    \right) +R_m^{(2)}
      \sin\left(\omega x   
    \right)\right)
    \end{array}
    $\\
    \hline
\end{tabular}
\subsection{欧拉方程}
$x^ny^{(n)}+p^1x^{n-1}y^{(n)}+\dots+p^{n-1}xy^{'}+p^ny=f\left(x\right)$\par
$x=e^t,t=\ln x$\par
$\frac{dy}{dx}=\frac{dy}{dt}\frac{dt}{dx}=\frac{1}{x}\frac{dy}{dt}$
$D\text{表示}\frac{d}{dt}$
$\left\{\begin{array}{l}
    xy^{'}  =Dy\\
    x^2y^{''}  =\left(D^2-D\right)y\\
    x^3y^{'''}  =\left(D^3-3D^2+2D\right)y\\
    \dots\\
    x^ky^{(k)}  =\left(D^n+c_1D^{n-1}+\dots +C_{n-1}D\right)y
\end{array}\right.$
\subsection{常系数线性微分方程组}


$\left\{\begin{array}{l}
    \frac{d^2x}{dt^2}+\frac{dy}{dt}-x=e^t,\\
    \frac{d^2y}{dt^2}+\frac{dx}{dt}+y=0,
\end{array}\right.
$\par
$\text{记}\frac{d}{dt}\text{为}D\Rightarrow
\left\{\begin{array}{l}
    \left(D^2-1\right)x+Dy=e^t\\
    Dx+\left(D^2+1\right)y=0
\end{array}\right.
$



\newpage


\section{向量代数与空间解析几何}

\subsection{向量及其线性运算}
\setcounter{page}{1}
\subsubsection{向量的概念}
大小,方向,\mb{向量},$\mha{AB}$\p
与起点无关,\mb{自由向量}\p
向量的大小,\mb{向量的模},$\mfa{\mha{AB}}$\mb{,单位向量,零向量}\p

$\mha{OA}=a,\mha{OB}=b,\angle AOB < \pi \mt{向量夹角},\ma{
    \mf{\widehat{a,b}} & =0  \mt{ or }  \pi,a\mt{与}b\mt{平行,同起点共线}\\
     & =\frac{\pi}{2},a\mt{与}b\mt{垂直}
}$

\subsubsection{向量的运算}

$\mb{向量的加减法}$\p
$\ma{c=a+b\mt{三角形法则}\\
    a+b=b+a\mt{交换}\\
    \mf{a+b}+c=a+\mf{b+c}\mt{分配}\\
    a-b=a+\mf{-b}\mt{负向量}\\
    \mfa{a \pm b}<\mfa{a}+\mfa{b}
}$\p

$\mb{向量与数的乘法}$\p
$\ma{
    \mfa{\lambda a}=\mfa{\lambda }\,\mfa{a}\\
    \lambda \mf{\mu a}= \mu \mf{ \lambda a}=\mf{\lambda \mu} a\\
    \mfa{\lambda \mf{\mu a}}= \mfa{\mu \mf{ \lambda a}}=\mfa{\mf{\lambda \mu} a}= \mfa{\lambda \mu} \,\mfa{a}\\
    \mf{\lambda+\mu}a= \lambda  a +  \mu a,\lambda\mf{a+b}=\lambda a+\lambda b
}$\p

$a \neq 0,a // b \Leftrightarrow \exists \mt{唯一实数}\lambda,b=\lambda a$\p

$\mb{空间直角坐标系}$\p
$\mfc{O;i,j,k}\mt{右手规则,卦限}$\p
$\mha{OM}=r=xi+yj+zk\mt{坐标式分解,分向量},r=\mf{x,y,z},M\mf{x,y,z}\mt{坐标},r \mt{为} M \mt{关于}O\mt{向径}$\p

$\mb{利用坐标向量的线性运算}$\p

$a=\mf{a_x,a_y,a_z},b=\mf{b_x,b_y,b_z}\left\{\ma{
    a+b & =\mf{a_x+b_x,a_y+b_y,a_z+b_z}\\
    a-b & =\mf{a_x-b_x,a_y-b_y,a_z-b_z}\\
    \lambda a & =\mf{\lambda a_x,\lambda a_y,\lambda a_z}
}\right.$

$\mb{向量的模,方向角,投影}$\p

$\q\q \mb{向量的模与两点间的距离}$\p
$r=\mf{x,y,z},\mfa{r}=\sqrt{x^2+ y^2+ z^2}$\p
$A=\mf{x_1 ,y_1, z_1},B=\mf{x_2, y_2, z_2},\mfa{AB}=\sqrt{\mf{x_2-x_1}^2+\mf{y_2-y_1}^2+\mf{z_2-z_1}^2 }$\p

$\q\q \mb{方向角与方向余弦}$\p
$r\mt{与三条坐标轴的夹角} \alpha, \beta, \gamma \mt{称为} r \mt{的方向角}$\p
$\mf{\cos \alpha ,\cos \beta,\cos \gamma}=\frac{1}{\mfa{r}}\mf{x,y,z},\cos * \mt{方向余弦}$

$\q\q \mb{向量在轴上的投影}$\p
$\left\{\ma{
    \mf{a}_u=\mfa{a}\cos \varphi\\
    \mf{a+b}_u=\mf{a}_u+\mf{b}_u=\mfa{a}\cos \varphi+\mfa{b}\cos \varphi\\
    \mf{\lambda a}_u=\lambda \mf{a}_u
}\right.$


\subsection{数量积 $\q$向量积$\q$ 混合积}

\subsubsection{兩向量的数量积}

$a \cdot b=\mfa{a}\,\mfa{b}\,\cos \theta \mt{數量積}$\p

$ \left\{ \ma{
    a\cdot a=\mfa{a}^2\\
    a,b \neq 0,\,a\cdot b=0 \Leftrightarrow a \perp b\\
    a\cdot b=b\cdot a \mt{交換}\\
    \mf{a+b} \cdot c=a\cdot c+ b\cdot c \mt{分配}\\
    \mf{\lambda a}\cdot b=\lambda \mf{a \cdot b} \mt{結合}
}\right.$\p

$\ma{a=\mf{a_x ,a_y ,a_z}\\b=\mf{b_x ,b_y, b_z}} \Rightarrow \ma{
    a\cdot b=a_xb_x+ a_yb_y+ a_zb_z\\
    \cos \theta=\frac{a\cdot b}{\mfa{a}\,\mfa{b}}=\frac{a_xb_x+ a_yb_y+ a_zb_z}{\sqrt{{a_x}^2 + {a_y}^2 + {a_z}^2}+\sqrt{{b_x}^2 + {b_y}^2 + {b_z}^2}}
}$\p


\subsubsection{兩向量的向量积}

$\mfa{c}=\mfa{a} \,\mfa{b}\sin \theta$\p
$c=a\times  b\mt{向量積}$\p

$\left\{\ma{
    a \times a =0\\
    a,b \neq 0, a \times b =0 \Leftrightarrow a // b \\
    a \times b =- b \times a\\
    \mf{a+b}\times c= a \times c+ b \times c\\
    \mf{\lambda a} \times b = a \times \mf{  \lambda b} =\lambda \mf{a \times  b}
}\right.$\p
$\ma{a=\mf{a_x ,a_y ,a_z}\\b=\mf{b_x ,b_y, b_z}} \Rightarrow \ma{a \times b &=\mf{ a_yb_z-a_zb_y, a_zb_x-a_xb_z,a_xb_y-a_yb_x}\\
    &=\mfa{\ma{
    i    &j    &k \\
    a_x  &a_y  &a_z\\
    b_x  &b_y  &b_z

    }}
}
$\p


\subsubsection{向量的混合积}
$\mf{a\times b} \cdot c=[abc],\mt{向量的混合积}$\p
$\mfa{[abc]}=\mfa{a} \,\mfa{b}\,\sin \mf{\widehat{a,b}} \mfa{c} \cos \mf{\widehat{a\times b,c}},\,\mt{a,b,c右手系模大于}0 $\p
$\ma{a=\mf{a_x ,a_y ,a_z}\\
b=\mf{b_x ,b_y, b_z}\\
c=\mf{c_x ,c_y, c_z}
} \Rightarrow \ma{
    [abc] &=\mf{a\times b} \cdot c\\
         &=\mfa{\ma{
        a_x  &a_y  &a_z\\
        b_x  &b_y  &b_z\\
        c_x  &c_y  &c_z\\
        }}
}
$\p

\subsection{平面及其方程}

\subsubsection{曲面方程与空间曲线方程的概念}

$F\mf{x,y,z}=0,\mt{曲面S上点满足方程,不在曲面S上点不满足方程}\Rightarrow\mt{曲面S的方程}$\p
$\left\{\ma{
    F\mf{x,y,z}=0\\
    G\mf{x,y,z}=0
}\right.\Rightarrow\mt{曲线C的方程}$

\subsubsection{平面的点法式方程}
向量垂直平面,平面的法线向量\p

$\ma{n=\mf{A,B,C}&\mt{法向量}\\ \mha{M_0M}=\mf{x-x_0,y-y_0,z-z_0}&\mt{平面上点}}\Rightarrow 
A\mf{x-x_0}+B\mf{y-y_0}+C\mf{z-z_0}=0\mt{平面的点法式方程}$

\subsubsection{平面的一般方程}
$Ax+By+Cz+D=0\mt{平面的一般方程}$\p
$\md{\frac{x}{a}+\frac{y}{b}+\frac{z}{c}=1}\mt{平面的截距式方程}$

\subsubsection{平面的夹角}
平面法向量的夹角,平面的夹角$\mf{0\leqslant \theta \leqslant \frac{\pi}{2}}$\p
$\varPi_1,n_1=\mf{A_1,B_1,C_1},\varPi_2,n_2=\mf{A_2,B_2,C_2},\cos \theta=\mfa{\mf{\widehat{n_1,n_2}}}=\md{\frac
{\mfa{A_1A_2+B_1B_2+C_1C_2}}{\sqrt{A_1^2+B_1^2+C_1^2}\sqrt{A_2^2+B_2^2+C_2^2}}}
\mt{平面夹角}$\p
$\ma{d&=\mfa{\mha{P_1P_0}}\,\mfa{\cos \theta}\\
    &=\frac{\mfa{\mha{P_1P_0} \times n}}{\mfa{n}}\\
    &=\frac{\mfa{A\mf{x_0-x}+B\mf{y_0-y}+C\mf{z_0-z}}}{\sqrt{A^2+B^2+C^2}}\\
    &=\frac{\mfa{Ax_0+By_0+Cz_0+D}}{\sqrt{A^2+B^2+C^2}}\\
}$


\subsection{空间直线及其方程}
\subsubsection{空间直线的一般方程}
$\left\{\ma{
    A_1x+B_1y+C_1z+D_1=0\\
    A_2x+B_2y+C_2z+D_2=0\\
}\right.$

\subsubsection{空间直线的对称式方程与参数方程}

向量平行直线,直线的方向向量\p

$\md{\ma{\mha{M_oM}&=\mf{x-x_0,y-y_0,z-z_0}\\
    S&=\mf{m,n,p}}\Rightarrow
  \ma{  \md{\frac{x-x_0}{m}=\frac{y-y_0}{n}=\frac{z-z_0}{p}=t},\mt{对称式,点向式方程}\\
  \left\{  \ma{
        x=x_0+mt\\
        y=y_0+nt\\
        z=z_0+pt\\
    }\right. \mt{参数方程}
    }
}$

\subsubsection{两直线夹角}
$L_1,s_1=\mf{A_1,B_1,C_1},L_2,s_2=\mf{A_2,B_2,C_2},\cos \theta=\mfa{\mf{\widehat{s_1,s_2}}}=\md{\frac
{\mfa{A_1A_2+B_1B_2+C_1C_2}}{\sqrt{A_1^2+B_1^2+C_1^2}\sqrt{A_2^2+B_2^2+C_2^2}}}
\mt{直线夹角}$\p


\subsubsection{直线与平面夹角}

直线和投影直线的夹角,直线与平面夹角\p

 
$\ma{\varphi=\mfa{\frac{\pi}{2}-\mf{\mha{s,n}}}\\s=\mf{m,n,p}\\n=\mf{A,B,C}} \Rightarrow \ma{\sin \varphi&=\frac{\mfa{Am+Bn+Cp}}{
    \sqrt{A^2+B^2+C^2}\sqrt{m^2+n^2+p^2}}\\
    &\frac{A}{m}=\frac{B}{n}=\frac{C}{p}\mt{垂直}
}$

\subsubsection{例}

\noindent
$\left.\ma{A_1x+B_1y+C_1z+D_1=0\q \mR{1}  \\
    A_2x+B_2y+C_2z+D_2=0\q \mR{2}
}\right\}L\Rightarrow 
A_1x+B_1y+C_1z+D_1+\lambda \mf{A_2x+B_2y+C_2z+D_2}=0\mt{过L平面束,不含$\mR{2}$}
$

\subsection{曲面及其方程}

\subsubsection{曲面研究的基本问题}

轨迹到方程,方程到形状\p
$x^2+y^2+z^2=R^2\mt{球面}$\p
$\mf{x-x_0}^2+\mf{y-y_0}^2+\mf{z-z_0}^2=R^2\mt{球面}$\p
$Ax^2+Ay^2+Az^2+Dx+Ey+Fz+G=0\mt{一般球面}$

\subsubsection{旋转曲面}
平面上曲线绕直线旋转一周,母线,轴,旋转曲面\p
$f\mf{y,z}=0 \Rightarrow f\mf{\pm \sqrt{x^2+y^2},z}=0\mt{圆台}$\p

相交直线L绕R旋转一周,交点顶点,夹角$\mf{0<\alpha<\frac{\pi}{2}}$半顶角\p
$z^2=a^2\mf{x^2+y^2}\mt{圆锥}$\p
$\frac{x^2}{a^2}-\frac{z^2}{b^2}=1\mt{双曲线} \Rightarrow \left.\ma{
    \frac{\mf{x^2+y^2}^2}{a^2}-\frac{z^2}{b^2}=1\mt{单叶双曲面}\\
    \frac{x^2}{a^2}-\frac{\mf{z^2+y^2}^2}{b^2}=1\mt{双叶双曲面}
}\right\}\mt{两种二次}$

\subsubsection{柱面}
$x^2+y^2=R^2\mt{圆柱面,准线(定曲线),母线}$\p
$\left.\ma{y^2=ax\mt{抛物柱面}\\
\frac{x^2}{a^2}+\frac{y^2}{b^2}=1\mt{椭圆柱面}\\
\frac{x^2}{a^2}-\frac{y^2}{b^2}=1\mt{双曲柱面}}\right\}\mt{三种二次}$


\subsubsection{二次曲面九}

$\frac{x^2}{a^2}+\frac{y^2}{b^2}=z^2\mt{椭圆锥面}$\p
$\frac{x^2}{a^2}+\frac{y^2}{b^2}=z\mt{椭圆抛物面}$\p
$\frac{x^2}{a^2}-\frac{y^2}{b^2}=z\mt{双曲抛物面}$\p
$\frac{x^2}{a^2}+\frac{y^2}{b^2}+\frac{z^2}{c^2}=1\mt{椭球面}$\p



\subsection{空间曲线及其方程}
\subsubsection{空间曲线一般方程}

$\left\{\ma{F\mf{x,y,z}=0\\
    G\mf{x,y,z}=0
}\right.$

\subsubsection{空间曲线参数方程}
$\left\{\ma{
    x=x\mf{t}\\
    y=y\mf{t}\\
    z=z\mf{t}
}\right. \Rightarrow \left\{\ma{
    x=a\cos \theta\\
    y=a\cos \theta\\
    z=b \theta
}\right.\mt{螺旋线},2\pi b\mt{螺距}$\p

\subsubsection*{空间曲线参数方程}
$\Gamma \left\{ \ma{
    x=\varphi \mf{t}\\
    y=\psi \mf{t}\\
    z=\omega \mf{t}\\
}\right.\Rightarrow 
 \left\{ \ma{
    x=\sqrt{\mfc{\varphi \mf{t}}^2+\mfc{\varphi \mf{t}}^2}\cos \theta\\
    y=\sqrt{\mfc{\varphi \mf{t}}^2+\mfc{\varphi \mf{t}}^2}\sin \theta
    z=\omega \mf{t},
}\right.\mt{曲面}
$\p
$\Gamma \left\{ \ma{
    x=a\sin \varphi\\
    y=0\\
    z=a\cos \varphi\\
}\right.\Rightarrow 
 \left\{ \ma{
    x=\sqrt{\mfc{a\sin \varphi}^2+\mfc{0}^2}\cos \theta=a\sin \varphi \cos \theta\\
    y=\sqrt{\mfc{a\sin \varphi}^2+\mfc{0}^2}\sin \theta=a\sin \varphi \sin \theta\\
    z=a\cos \varphi,
}\right.\mt{球面}
$
\subsubsection{空间曲线在坐标面投影}
$H\mf{x,y}=0$投影柱面\p
$\left\{ \ma{
    H\mf{x,y}=0\\
    z=0
}\right.$投影曲线

\newpage
\section{多元函数微分及其应用}

\subsection{多元函数的基本概念}
\subsubsection{平面點集 $\q ^*n$維空間 }
\mb{平面點集}\p
$R^2$坐標平面\p
$C=\mfb{\mf{x,y}|x^2+y^2<r^2}$平面點集\p
$U\mf{P_0,\delta}=\mfb{P|PP_0<\delta}$鄰域\p
$\mathring{U}\mf{P_0,\delta}=\mfb{P|0<PP_0<\delta}$去心鄰域\p


$\ma{
    P\in R^2,\\
    E \subset R^2
}\ma{ \exists U\mf{P} ,U\mf{P}\subset E ,P\mt{為}E\mt{內點}\\
        \exists U\mf{P} ,U\mf{P} \cap  E=\varnothing  ,P\mt{為}E\mt{外點}\\
        \left.\ma{
        P=P_1 \cap P_2\\
        \exists U\mf{P_1} ,U\mf{P_1}\subset E\\
        \exists U\mf{P_2} ,U\mf{P_2} \cap  E=\varnothing
        }\right\},P\mt{為}E \mt{邊界點}
}$\tp{
    \tpo{a}{(0,0)};
    \tpo{b}{(0.707,0.707)};
    \tpo{c}{(1.4,.4)};
    % [rotate = 60]
    \draw [rotate = 45] (a) ellipse (1 and .5);
    \foreach \p in {a,b,c} \fill [opacity = 0.75] (\p) circle (4pt);
    \foreach \p in {a,b,c} \fill [opacity = 1] (\p) circle (1pt);

}\p
$E\mt{邊界點的全體,為}E \mt{邊界點},\partial E$\p
$\forall \delta>0,\mathring{U}\mf{P,\delta}\mt{內總有}E\mt{中的點},P\mt{為}E\mt{聚點(內點和邊界)}$\p
開集,內點;閉集,$\partial E \subset E$;連通集,任意點連線仍在集合;\p
區域,連通開集;閉區域,區域和邊界;\p 
有界集,$\forall E \subset R^2,if \q \exists r>0,E \subset U\mf{O,r},then \q \mt{有界集} $;無界集,不是有界集\p

\mb{n維空間}\p
$R^n=\mfb{\mf{x_1,x_2,\cdots,x_n}|x_i \in R ,i=1,2,\cdots,n}=x \mt{集合}$\p
$\left.\ma{
    x=\mf{x_1,x_2,\cdots,x_n} \in R^n \\
    y=\mf{y_1,y_2,\cdots,y_n} \in R^n \\
    a=\mf{a_1,a_2,\cdots,a_n} \in R^n \\
    \lambda \in R
}\right\} \Rightarrow \ma { 
    \left.\ma{
        x+y &=\mf{x_1+y_1,x_2+y_2,\cdots,x_n+y_n}\\
        \lambda x &= \mf{\lambda x_1,\lambda x_2,\cdots,\lambda x_n}\\
    } \right\} R^n \mt{中線性運算}\\
    \rho\mf{x,y} =\sqrt{ \mf{x_1-y_1}^2+\mf{x_2-y_2}^2+\cdots+\mf{x_n-y_n}^2 }\\
    ||x||= \rho\mf{x,0} =\sqrt{ {x_1}^2+{x_2}^2+\cdots+{x_n}^2 }\\
    ||x-a||\rightarrow 0\mt{,變元趨于固定元,記作}x\rightarrow a\\
    \mt{某正數}\delta>0,\mt{點集}U\mf{a,\delta}=\mfb{x|x\in R^n,\rho\mf{x,a}<\delta},a\mt{的}\delta\mt{鄰域}
}
$\p
定義線性預算的集合,n維空間\p


\subsubsection{多元函數的概念}

$\ma{D \subset R^2,\mt{映射}f:D\rightarrow R ,\mt{稱為二元函數,}\\
\mt{記為}z=f\mf{x,y},\mf{x,y}\in D}
\Rightarrow \ma{
    D \mt{定義域}\\
    x,y\mt{自變量}\\
    z\mt{因變量}\\
    \mt{函數值}f\mf{x,y}\mt{全體},f\mf{D}=\mfb{z|z=f\mf{x,y},\mf{x,y}\in D}\mt{值域}\\
}$\p


$\ma{D \subset R^n,\mt{映射}f:D\rightarrow R ,\mt{稱為}n\mt{元函數(}n\geqslant2\mt{多元函數),}\\
\mt{記為}z=f\mf{x_1,x_2,\cdots,x_n}=f\mf{\mb{x}},\mb{x}\mf{x_1,x_2,\cdots,x_n}\in D}$\p

$\mt{多元函數} \mu=f\mf{\mb{x}},\mt{有意義的變元}\mb{x}\mt{的點集,自然定義域}$

$\mt{空間點集}\mfb{\mf{x,y,z}|z=f\mf{x,y},\mf{x,y}\in D},\mt{二元函數} z=f\mf{x,y}\mt{的圖形}$

\subsubsection{多元函數的極限}

$\ma{
    f\mf{x,y}\mt{定義域}D,\\
    P_0\mf{x_0,y_0}\mt{是}D\mt{聚點}
}\ma{ if  \q 
 &\exists A ,\forall \varepsilon>0 ,\exists \delta>0,\\
 &when \,\mt{點}P\mf{x_0,y_0} \in D \cap \mathring{U}\mf{P_0,\delta},\\
 &\mfa{f\mf{P}-A}=\mfa{f\mf{x,y}-A}<\varepsilon}
\ma{then \q  &A\mt{稱為函數} f\mf{x,y} \mt{在}\mf{x,y} \rightarrow \mf{x_0,y_0}\mt{極限,}\\ 
    &\mt{記為} \md{\lim_{\mf{x,y} \rightarrow \mf{x_0,y_0}}f\mf{x,y}=A} },\\
    \mt{任意方向趨近}
$\p

\subsubsection{多元函數的連續性}
 
$f\mf{P}=f\mf{x,y}\mt{定義域}D,P_0\mf{x_0,y_0}\mt{為}D\mt{聚點},P_0\in D,\ma{if \q \md{\lim_{\mf{x,y} \rightarrow \mf{x_0,y_0}}f\mf{x,y}=f\mf{x_0,y_0}},\\ then \q f\mf{x,y}\mt{在}P_0\mf{x_0,y_0}\mt{連續}}$\p

$f\mf{x,y}\mt{定義域}D,D\mt{內每一點都是聚點},f\mf{x,y}\mt{在}D\mt{內每一點連續},f\mf{x,y}\mt{在}D\mt{內連續}$\p

$f\mf{x,y}\mt{定義域}D,P_0\mf{x_0,y_0}\mt{為}D\mt{聚點},if \q f\mf{x,y}\mt{在}P_0\mf{x_0,y_0}\mt{不連續},then \q P_0\mf{x_0,y_0}\mt{為}f\mf{x,y}\mt{間斷點}$\p


常數,不同自變量的一元基本初等函數,有限次四則運算和復合運算,\mb{多元初等函數}\p
一切多元初等函數在\mb{定義區域}(定義域內的區域或閉區域)連續$\rightarrow \md{\lim_{P \rightarrow P_0}f\mf{P}=f\mf{P_0}}$\p

$\mt{性質}\ma{
    \mt{有界閉區域D,多元連續函數,D上有界,取得最大值,最小值} & \mf{\mb{最值}}\\
    \mt{有界閉區域D,多元連續函數,能取得介於最大值和最小值間的任何值} & \mf{\mb{介值}}\\
    \mt{有界閉區域D,多元連續函數,D上一致連續} & \mf{\mb{一致連續}}\\
}$\p



\subsection{偏導數}

\subsubsection{偏導數的定義及其計算法}

$\ma{f\mf{x,y},P_0\mf{x_0,y_0},\mf{x,y} \in U\mf{P_0,\delta},\\
when  \q  y=y_0 ,x= x_0+\Delta x,\\
\mt{函數增量} f\mf{x_0+\Delta x,y_0}-f\mf{x_0,y_0}}
\ma{
    if \q  \exists \md{\lim_{\Delta x \rightarrow 0}\frac{f\mf{x_0+\Delta x,y_0}-f\mf{x_0,y_0}}{\Delta x}}=A,\\
    then \q  A \mt{為} f\mf{x,y} \mt{在} \mf{x_0,y_0} \mt{對} x \mt{的偏導數,}\\
    \mt{記為} \md{  \left. \frac{\partial f}{\partial x} \right|_{\tiny\ma{x=x_0\\y=y_0}}},f_x\mf{x_0,y_0}
}$\p

$\left\{\ma{
    \md{  \left. \frac{\partial f}{\partial x} \right|_{\tiny\ma{x=x_0\\y=y_0}}}=\md{\lim_{\Delta x \rightarrow 0}\frac{f\mf{x_0+\Delta x,y_0}-f\mf{x_0,y_0}}{\Delta x}}\\
    \md{  \left. \frac{\partial f}{\partial y} \right|_{\tiny\ma{x=x_0\\y=y_0}}}=\md{\lim_{\Delta y \rightarrow 0}\frac{f\mf{x_0,y_0+\Delta x}-f\mf{x_0,y_0}}{\Delta y}}

}\right.$\p
$f_x\mf{x,y},f_y\mf{x,y}\mt{偏導函數,偏導數}$\p
偏導數記號為整體,不能看成微分的商\p
(一元可導連)各偏導存在,不一定連續\p


\subsubsection{高階偏導數}

$z=\ma{f\mf{x,y}\mt{的偏导数}\frac{\partial z}{\partial x}=f_x\mf{x,y},\frac{\partial z}{\partial y}=f_y\mf{x,y},\\
\mt{都是}x,y\mt{的函数,}} \Rightarrow
\ma{\mt{若偏导数的偏导仍存在,则称为}f\mf{x,y}\mt{的二阶偏导数}\\
\frac{\partial}{\partial x}\mf{\frac{\partial z}{\partial x}}=\frac{\partial^2z}{\partial x^2}=f_{xx}\mf{x,y}  \\
\frac{\partial}{\partial x}\mf{\frac{\partial z}{\partial y}}=\frac{\partial^2z}{\partial y \partial x}=f_{yx}\mf{x,y} \mt{(混合偏导数)} \\
\frac{\partial}{\partial y}\mf{\frac{\partial z}{\partial x}}=\frac{\partial^2z}{\partial x \partial y}=f_{xy}\mf{x,y} \mt{(混合偏导数)} \\
\frac{\partial}{\partial y}\mf{\frac{\partial z}{\partial y}}=\frac{\partial^2z}{\partial y^2}=f_{yy}\mf{x,y}  \\
}$\p

\qa

$if \q  f\mf{x,y}\mt{的二阶偏导数}, \frac{\partial^2z}{\partial y \partial x} ,\frac{\partial^2z}{\partial x \partial y}\mt{在D连续},then \q  \mf{x,y} \in D \frac{\partial^2z}{\partial y \partial x} =\frac{\partial^2z}{\partial x \partial y}$

$\left.\ma{z=\ln \sqrt{x^2+y^2}, \frac{\partial^2 z}{\partial x^2}+\frac{\partial^2 z}{\partial y^2}=0\\
    u=\frac{1}{\sqrt{x^2+y^2+z^2}},\frac{\partial^2 u}{\partial x^2}+\frac{\partial^2 u}{\partial y^2}+\frac{\partial^2 u}{\partial z^2}=0\\
}\right\}\mt{拉普拉斯方程}$


\subsection{全微分}
\subsubsection{全微分定義}
$\left\{\ma{f\mf{x+\Delta x,y}-f\mf{x,y} \approx f_x\mf{x,y}\Delta x\\
    f\mf{x,y+\Delta y}-f\mf{x,y} \approx f_y\mf{x,y}\Delta y\\
}\right. \q  \mt{偏增量,偏微分}$\p

$\Delta z=f\mf{x+ \Delta x,y+\Delta y}-f\mf{x,y} \q \mt{全增量}$\p

$z=f\mf{x,y},\mt{在}\mf{x,y}\mt{的某邻域有定义},\ma{if \q  &\Delta z=f\mf{x+ \Delta x,y+\Delta y}-f\mf{x,y}=A\Delta x +B\Delta  y+ o\mf{\rho},\\
then \q 
 &\rho=\sqrt{\mf{\Delta x}^2+\mf{\Delta y}^2},A,B\mt{不依赖}\Delta x ,\Delta y\\
 &f\mf{x,y}\mt{在}\mf{x,y}\mt{可微分},\\
 &A\Delta x +B\Delta  y ,\mt{称为函数}f\mf{x,y}\mt{全微分},\\
 &\mt{记作}dz=A\Delta x +B\Delta  y
}$\p

多元函数在区域D内个点处都可微分,函数在D内可微分\p
多元函数在\mb{点P}可微分,函数在该点连续\p


$if \q f\mf{x,y}\mt{在点} \mf{x,y} \mt{可微分},\ma{
    then \q  \mt{函数}f\mt{x,y}  \mt{在点}\mf{x,y}\mt{偏导数} \frac{\partial z}{\partial x},\frac{\partial z}{\partial y} \mt{必存在}\\
    \mt{函数}f\mf{x,y}  \mt{在点}\mf{x,y}\mt{全微分} dz=\frac{\partial z}{\partial x}dx+\frac{\partial z}{\partial y}dy
}$\p

$z=f\mf{x,y}\mt{的偏導數}\frac{\partial z}{\partial x},\frac{\partial z}{\partial y}\mt{在點}\mf{x,y}\mt{連續},\mt{函數在該點可微分} $\p

$\mt{微分疊加原裡},u=\mf{x,y,z},du=\frac{\partial u}{\partial x}dx+\frac{\partial u}{\partial y}dy+\frac{\partial u}{\partial z}dz$\p


\subsubsection{全微分在近似計算中的應用}

$g=\frac{4\pi^2l}{T^2},\Delta g\leqslant 4\pi^2\mf{\frac{1}{T^2}\delta_l  +\frac{2l}{T^3}\delta_T}$\p


\subsection{多元復合函數的求導法則}

$if \ma{
    u=\varphi\mf{t},v=\psi\mf{t},\mt{在}t\mt{可導},\\
    z=f\mf{u,v}\mt{在}\mf{u,v}\mt{具有連續偏導}
},then \q \mt{復合函數}z=f\mf{\varphi\mf{t},\psi\mf{t}}\mt{在}t\mt{點可導,且}\md{\frac{dz}{dt}=\frac{\partial z}{\partial u}\frac{du}{dt}+\frac{\partial z}{\partial v}\frac{dv}{dt}}\mt{全導數}$\p
\qa

$if \q u=\varphi\mf{t},v=\psi\mf{t} ,w=\omega\mf{t},\mt{在}t\mt{可導} , z=f\mf{\varphi\mf{t},then \q \psi\mf{t}+\omega\mf{t}},\md{\frac{dz}{dt}=\frac{\partial z}{\partial u}\frac{du}{dt}+\frac{\partial z}{\partial v}\frac{dv}{dt}+\frac{\partial z}{\partial w}\frac{dw}{dt}}$\p
\qa

$if \ma{
    u=\varphi\mf{x,y},v=\psi\mf{x,y},\\
    \mt{在}\mf{x,y}\mt{具有對}x,y\mt{的偏導數},\\
    z=f\mf{u,v}\mt{在}\mf{u,v}\mt{具有連續偏導}
},then \q 
\ma{
    \mt{復合函數}z=f\mf{\varphi\mf{x,y},\psi\mf{x,y}}\mt{在}(x,y)\mt{點兩偏導都存在,且}\\
    \frac{\partial z}{\partial x}=\frac{\partial z}{\partial u}\frac{\partial u}{\partial x}+\frac{\partial z}{\partial v}\frac{\partial v}{\partial x}\\
    \frac{\partial z}{\partial y}=\frac{\partial z}{\partial u}\frac{\partial u}{\partial y}+\frac{\partial z}{\partial v}\frac{\partial v}{\partial y}
}$\p
\qa

$if \q \ma{ u=\varphi\mf{x,y},v=\psi\mf{x,y} ,w=\omega\mf{x,y},\\
\mt{在}\mf{x,y}\mt{點兩偏導都存在} \\
 z=f\mf{\varphi\mf{x,y},\psi\mf{x,y}+\omega\mf{x,y}},},then \q
\ma{ \frac{\partial z}{\partial x}=\frac{\partial z}{\partial u}\frac{\partial u}{\partial x}+\frac{\partial z}{\partial v}\frac{\partial v}{\partial x}+\frac{\partial z}{\partial w}\frac{\partial w}{\partial x}\\
 \frac{\partial z}{\partial y}=\frac{\partial z}{\partial u}\frac{\partial u}{\partial y}+\frac{\partial z}{\partial v}\frac{\partial v}{\partial y}+\frac{\partial z}{\partial w}\frac{\partial w}{\partial y}
}
$\p
\qa


$if \q \ma{u& =\varphi\mf{x,y}\mt{在}\mf{x,y}\mt{點兩偏導都存在},\\
v &=\psi\mf{y} \mt{在}y\mt{點可導} ,\\
z &=f\mf{u,v} \mt{在}\mf{u,v}\mt{具有連續偏導}
} \ma{then \q &z=f\mf{\varphi\mf{x,y},\psi\mf{y} }\mt{在}\mf{x,y}\mt{點兩偏導都存在,且}\\
    &\frac{\partial z}{\partial x}=\frac{\partial z}{\partial u}\frac{\partial u}{\partial x}\\
    &\frac{\partial z}{\partial y}=\frac{\partial z}{\partial u}\frac{\partial u}{\partial y}+\frac{\partial z}{\partial v}\frac{\partial v}{\partial y} 
}$\p
\qa

$if z=f\mf{u,x,y}=f\mfc{\varphi\mf{x,y},x,y} \mt{有}x,y\mt{的偏導數},then \q \ma{
    &\frac{\partial z}{\partial x}=\frac{\partial f}{\partial u}\frac{\partial u}{\partial x}+ \frac{\partial f}{\partial x}\\
    &\frac{\partial z}{\partial y}=\frac{\partial f}{\partial u}\frac{\partial u}{\partial y}+\frac{\partial f}{\partial y} 

}$\p


$f\mf{u,v},\ma{
    f_{1}^{'}=f_u\mf{u,v}\\
    f_{2}^{'}=f_v\mf{u,v}\\
    f_{11}^{'}=f_{uu}\mf{u,v}\\
    f_{12}^{'}=f_{uv}\mf{u,v}\\
    f_{21}^{'}=f_{vu}\mf{u,v}\\
    f_{22}^{'}=f_{vv}\mf{u,v}\\
}$\p

$z=f\mf{u,v}=f\mf{\varphi\mf{x,y},\psi\mf{x,y}} ,\q \ma{dz &=\frac{\partial z}{\partial u}du+\frac{\partial z}{\partial v}dv\\
    &=\frac{\partial z}{\partial x}dx+\frac{\partial z}{\partial y}dy\\
    &=\mf{\frac{\partial z}{\partial u}\frac{\partial u}{\partial x}+\frac{\partial z}{\partial v}\frac{\partial v}{\partial x}}dx+
    \mf{\frac{\partial z}{\partial u}\frac{\partial u}{\partial y}+\frac{\partial z}{\partial v}\frac{\partial v}{\partial y}}dy
}$


\subsection{隱函數求導公式}

\subsubsection{一個方程}
$if \q \ma{
    P_0\mf{x_0,y_0},\mf{x,y} \in U\mf{P_0,\delta}\mt{時}\\
    F\mf{x,y}\mt{具有連續偏導},\\
    F\mf{x_0,y_0}=0,F_y\mf{x_0,y_0}\neq 0
},then \q \ma{
    F\mf{x,y}=0,\mt{在} \mf{x,y} \in U\mf{P_0,\delta}\mt{時,}\\
    \mt{能確定唯一連續,連續導數的函數}y=f\mf{x},\\
    y_0=f\mf{x_0},\frac{dy}{dx}=-\frac{F\mf{x}}{F\mf{y}}
}$\p
\qa 

$if \q \ma{
    P_0\mf{x_0,y_0,z_0},\mf{x,y,z} \in U\mf{P_0,\delta}\mt{時}\\
    F\mf{x,y,z}\mt{具有連續偏導},\\
    F\mf{x_0,y_0,z_0}=0,F_z\mf{x_0,y_0,z_0}\neq 0
},then \q \ma{
    F\mf{x,y,z}=0,\mt{在} \mf{x,y,z} \in U\mf{P_0,\delta}\mt{時,}\\
    \mt{能確定唯一連續,連續導數的函數}z=f\mf{x,y},\\
    z_0=f\mf{x_0,y_0},\frac{\partial z}{\partial x}=-\frac{F\mf{x}}{F\mf{z}},\frac{\partial z}{\partial y}=-\frac{F\mf{y}}{F\mf{z}}
}$



\subsubsection{方程組的情形}


$if \q \ma{
    P_0\mf{x_0,y_0,u_0,v_0},\mf{x,y,u,v} \in U\mf{P_0,\delta}\mt{時}\\
    F\mf{x,y,u,v}, G\mf{x,y,u,v}\mt{具有連續偏導},\\
    F\mf{x_0,y_0,u_0,v_0}=0,G\mf{x_0,y_0,u_0,v_0}=0,\\
    J=\mfa{\ma{
        \frac{\partial F}{\partial u}  &\frac{\partial F}{\partial v}\\
        \frac{\partial G}{\partial u}  &\frac{\partial G}{\partial v}
    }}\neq 0\left(\ma{\mt{雅可比式}\\ \mt{偏导数组成的函数行列式}}\right)
}then \q \ma{
    F\mf{x,y,u,v}=0,G\mf{x,y,u,v}=0,\\
    \mt{在} \mf{x,y,u,v} \in U\mf{P_0,\delta}\mt{時,}\\
    \mt{能確定唯一組連續,連續偏導數的函數},\\
    u=f\mf{x,y},v=f\mf{x,y}\\
    u_0=f\mf{x_0,y_0},v_0=f\mf{x_0,y_0}\\
    \frac{\partial u}{\partial x}=-\frac{1}{J}\frac{\partial\mf{F,G}}{\partial\mf{x,v}}  \\
    \frac{\partial u}{\partial y}= -\frac{1}{J}\frac{\partial\mf{F,G}}{\partial\mf{y,v}} \\
    \frac{\partial v}{\partial x}=-\frac{1}{J}\frac{\partial\mf{F,G}}{\partial\mf{u,x}}  \\
    \frac{\partial v}{\partial y}= -\frac{1}{J}\frac{\partial\mf{F,G}}{\partial\mf{u,y}} \\
}$

$\left\{ \ma{
    F_x+F_u\frac{\partial u}{\partial x}+F_v\frac{\partial v}{\partial x}=0\\
    G_x+G_u\frac{\partial u}{\partial x}+G_v\frac{\partial v}{\partial x}=0\\
}\right.\Rightarrow \ma{
    \frac{\partial u}{\partial x}=\frac{\mfa{\ma{-F_x & F_v\\-G_x & G_v}}}{\mfa{\ma{F_u & F_v\\G_u & G_v}}}=-\frac{\frac{\partial\mf{F,G}}{\partial\mf{x,v}}}{\frac{\partial\mf{F,G}}{\partial\mf{u,v}}}=-\frac{1}{J}\frac{\partial\mf{F,G}}{\partial\mf{x,v}}\\
    \frac{\partial v}{\partial x}=\frac{\mfa{\ma{F_u & -F_x\\G_u & -G_x}}}{\mfa{\ma{F_u & F_v\\G_u & G_v}}}=-\frac{\frac{\partial\mf{F,G}}{\partial\mf{u,x}}}{\frac{\partial\mf{F,G}}{\partial\mf{u,v}}}=-\frac{1}{J}\frac{\partial\mf{F,G}}{\partial\mf{u,x}}\\
}\mt{证明或推导}$


\qa

$\ma{
dudv,x=\varphi\mf{u,v},y=\psi\mf{u,v},\\
dx=\frac{\partial \varphi}{\partial u}du+\frac{\partial \varphi}{\partial v}dv\\
dy=\frac{\partial \psi}{\partial u}du+\frac{\partial \psi}{\partial v}dv
}  \rightarrow \ma{
\mfa{\mf{dx,dy}|_{du=0}\times\mf{dx,dy}|_{dv=0}}\\
=\mfa{\mf{\frac{\partial \varphi}{\partial u}du,\frac{\partial \psi}{\partial u}du}\times \mf{\frac{\partial \varphi}{\partial v}dv,\frac{\partial \psi}{\partial v}dv}}\\
=\mfa{\frac{\partial \varphi}{\partial u}\cdot\frac{\partial \psi}{\partial v}-  \frac{\partial \psi}{\partial u} \cdot \frac{\partial \varphi}{\partial v}  }dudv\\
=\mfa{\ma{
    \frac{\partial \varphi}{\partial u} & \frac{\partial \varphi}{\partial v}\\
    \frac{\partial \psi}{\partial u} & \frac{\partial \psi}{\partial v}
}}dudv\\
=\frac{\partial\mf{\varphi,\psi}}{\partial \mf{u,v}}dudv
} \mt{二重积分换元和雅可比}$




\subsection{多元函数微分学的几何应用}

\subsubsection{一元向量值函数及其导数}

$\ma{\Gamma \left\{\ma{x=\varphi\mf{t},\\y=\psi\mf{t},\\z=\omega\mf{t}}\right.t\in \mfc{\alpha,\beta}\\
    \mt{記}r=xi+yj+zk,f\mf{t}=\varphi\mf{t}i+\psi\mf{t}j+\omega\mf{t}k,\\
    r=f\mf{t},t\in \mfc{\alpha,\beta}

}$\p

$\mt{數集}D \subset R,\mt{映射} f:D\rightarrow R^n\mt{為一元向量值函數,記為}r=f\mf{t},t\in D$\p


$t \in U\mf{t_0},\mt{向量值函数}f\mf{t},if \q \exists r_0,\forall \varepsilon>0,\exists \delta,when \q 0<\mfa{t-t_0}<\delta,\mfa{f\mf{t}-r_0}<\varepsilon,then \q\lim_{t \rightarrow t_0}f\mf{t}=r_0$\p
$t \in u\mf{t_0},\mt{向量值函数}f\mf{t},if \q \lim_{t \rightarrow t_0}f\mf{t}=f\mf{t_0},f\mf{t}\mt{在}t_0\mt{连续}$\p
$t \in D,,\mt{向量值函数}f\mf{t},if \q D_1 \subset D,D_1\mt{内每点连续},D_1\mt{上连续}$\p
$t \in U\mf{t_0},\mt{向量值函数}f\mf{t},if \q \exists \lim_{\Delta t \rightarrow 0}\frac{\Delta r}{\Delta t}=\lim_{\Delta t \rightarrow 0}\frac{f\mf{t_0+\Delta t}-f\mf{t_0}}{\Delta t},then \q \ma{\mt{极限向量为}r=f\mf{t}\\\mt{在}t_0\mt{导数或导向量,}\\ \mt{记为}f^{'}\mf{t_0},\frac{dr}{dt}|_{t=t_0}}$\p

$f^{'}\mf{t_0}=f_1^{'}\mf{t_0}i+f_2^{'}\mf{t_0}j+f_3^{'}\mf{t_0}k$\p
$\mt{向量值函数}u\mf{t},v\mf{t},\mt{数量值函数}\varphi\mf{t} \Rightarrow\ma{
    \frac{d}{dt}C=0\\
    \frac{d}{dt}\mfc{cu\mf{t}}=cu^{'}\mf{t}\\
    \frac{d}{dt}\mfc{u\mf{t} \pm v\mf{t}}=u^{'}\mf{t}+v^{'}\mf{t}\\
    \frac{d}{dt}\mfc{\varphi\mf{t}  v\mf{t}}=\varphi^{'}\mf{t}v\mf{t}+\varphi\mf{t}v^{'}\mf{t}\\
    \frac{d}{dt}\mfc{u\mf{t} \cdot  v\mf{t}}=u^{'}\mf{t}\cdot v\mf{t}+u\mf{t}\cdot v^{'}\mf{t}\\
    \frac{d}{dt}\mfc{u\mf{t} \times  v\mf{t}}=u^{'}\mf{t}\times v\mf{t}+u\mf{t}\times v^{'}\mf{t}\\
    \frac{d}{dt}u\mfc{\varphi\mf{t}}=\varphi^{'}\mf{t}u^{'}\mfc{\varphi\mf{t}}
}$\p

$f^{'}\mf{t_0}=\lim_{t \rightarrow t_0}\frac{\Delta r}{\Delta t}\mt{与曲线相切}$

\subsubsection{空间曲线的切线与法平面}

$\Gamma \left\{\ma{
    x=\varphi\mf{t},\\
    y=\psi\mf{t},\\
    z=\omega\mf{t},\\
}\right. \q t \in \mfc{\alpha,\beta}\Rightarrow \ma{
    \frac{x-x_0}{\varphi^{'}\mf{t}}=\frac{y-y_0}{\psi^{'}\mf{t}}=\frac{z-z_0}{\omega^{'}\mf{t}}\\
    \varphi^{'}\mf{t}\mf{x-x_0}+\psi^{'}\mf{t}\mf{y-y_0}+\omega^{'}\mf{t}\mf{z-z_0}=0
}$\p

$\Gamma \left\{\ma{
    y=\psi\mf{x},\\
    z=\omega\mf{x},\\
}\right.\Rightarrow 
\left\{\ma{
    x=x,\\
    y=\psi\mf{x},\\
    z=\omega\mf{x},\\
}\right. \Rightarrow \ma{
    \frac{x-x_0}{1}=\frac{y-y_0}{\psi^{'}\mf{t}}=\frac{z-z_0}{\omega^{'}\mf{t}}\\
     \mf{x-x_0}+\psi^{'}\mf{t}\mf{y-y_0}+\omega^{'}\mf{t}\mf{z-z_0}=0
}$\p


$\ma{\Gamma \left\{ 
   \ma{ F\mf{x,y,z}=0\\
    G\mf{x,y,z}=0}
\right.\\
\frac{\partial \mf{F,G}}{\partial \mf{y,z}}|_{\mf{x_0,y_0,z_0}}\neq 0} \Rightarrow \left\{\ma{
    y=\psi\mf{x},\\
    z=\omega\mf{x},\\
}\right.\Rightarrow \ma{\left\{ 
    \ma{ 
        F\mf{x,\psi\mf{x},\omega\mf{x}}=0\\
        G\mf{x,\psi\mf{x},\omega\mf{x}}=0}
 \right.\\
 \left\{ 
    \ma{ 
        \da{F}{x}+\da{F}{y}\db{y}{x}+\da{F}{z}\db{z}{x}=0\\
        \da{G}{x}+\da{G}{y}\db{y}{x}+\da{G}{z}\db{z}{x}=0}
 \right.\\
 \left\{ \ma{\db{y}{x}=\psi^{'}\mf{x}=-\frac{1}{J}\da{\mf{F,G}}{\mf{x,z}}=-\frac{\da{\mf{F,G}}{\mf{x,z}}}{\da{\mf{F,G}}{\mf{y,z}}}=-\frac{\mfa{\ma{ \da{F}{x}  &\da{F}{z} \\ \da{G}{x}  &\da{G}{z} }}}{\mfa{\ma{ \da{F}{y}  &\da{F}{z} \\ \da{G}{y}  &\da{G}{z} }}}\\
    \db{z}{x}=\omega^{'}\mf{x}=-\frac{1}{J}\da{\mf{F,G}}{\mf{y,x}}=-\frac{\da{\mf{F,G}}{\mf{y,x}}}{\da{\mf{F,G}}{\mf{y,z}}}=-\frac{\mfa{\ma{ \da{F}{y}  &\da{F}{x} \\ \da{G}{y}  &\da{G}{x} }}}{\mfa{\ma{ \da{F}{y}  &\da{F}{z} \\ \da{G}{y}  &\da{G}{z} }}}\\
 }\right.
 } $



 \subsubsection{曲面的切平面与法线}
$\ma{\sum  F\mf{x,y,z}=0\\
\mt{任意曲线}\Gamma \left\{\ma{
    x=\varphi\mf{t},\\
    y=\psi\mf{t},\\
    z=\omega\mf{t},\\
} \right. \mf{\alpha <t<\beta},\\
t=t_0\mt{对应}M\mf{x_0,y_0,z_0}\\
\mt{曲线}\Gamma \mt{在曲面}\sum \mt{上,过}M\mt{点}\\
F\mfc{\varphi\mf{t},\psi\mf{t},\omega\mf{t}}=0,\\
F_x|_M\varphi^{'}\mf{t_0} +F_y|_M\psi^{'}\mf{t_0}+F_z|_M\omega^{'}\mf{t_0}\\
=\mf{F_x|_M,F_z|_M,F_z|_M}\cdot \mf{\varphi^{'}\mf{t_0},\psi^{'}\mf{t_0},\omega^{'}\mf{t_0}}\\
=n\cdot s=0,\\
\mt{任一过}M\mt{曲线切线垂直过}M\mt{向量}n,\mt{切线共面,称为曲面切平面}
}\Rightarrow \ma{
    F_x|_M\mf{x-x_0}+F_z|_M\mf{y-y_0}+F_z|_M\mf{z-z_0}=0\\
    \frac{\mf{x-x_0}}{F_x|_M}=\frac{\mf{y-y_0}}{F_y|_M}=\frac{\mf{z-z_0}}{F_z|_M}
}$\p
\qa

$z=f\mf{x,y}\Rightarrow F\mf{x,y,z}=f\mf{x,y}-z=0\Rightarrow \ma{
    F_x|_M=f_x\mf{x_0,y_0},F_y|_M=f_y\mf{x_0,y_0},F_z|_M=-1\\
    \mt{记}K=\sqrt{f_x^2+f_y^2+1},\cos \gamma >0,\\
    \mt{方向余弦}\mf{\cos \alpha,\cos \beta,\cos \gamma}=\mf{\frac{-f_x}{K},\frac{-f_y}{K},\frac{1}{K}}\\
    f_x\mf{x_0,y_0}\mf{x-x_0}+f_y\mf{x_0,y_0}\mf{y-y_0}-\mf{z-z_0}=0\\
    \frac{\mf{x-x_0}}{f_x\mf{x_0,y_0}}=\frac{\mf{y-y_0}}{f_y\mf{x_0,y_0}}=\frac{\mf{z-z_0}}{-1}
}$



\subsection{方向导数与梯度}

\subsubsection{方向导数}
$\ma{\mt{单位向量}e_l=\mf{\cos \alpha,\cos \beta},\\
\mt{过}\mf{x_0,y_0}\mt{射线}l \fcz{x=x_0+t\cos \alpha \\y=y_0+t\cos \beta} \mf{t\geqslant 0},\\
z=f\mf{x,y}} \Rightarrow \ma{\da{f}{l}|_{\mf{x_0,y_0}}=\md{\lim_{t \rightarrow 0^+}}\frac{f\mf{x_0+t\cos \alpha,y_0+t\cos \beta}-f\mf{x_0,y_0}}{t}\\
\mt{方向导数}}
$\p

$f\mf{x,y}\mt{在}P_0\mf{x_0,y_0}\mt{可微分,该点任意方向,方向导数存在,}\da{f}{l}|_{\mf{x_0,y_0}}=f_x\mf{x_0,y_0}\cos \alpha+f_y\mf{x_0,y_0}\cos \beta$\p
$f\mf{x,y,z},\da{f}{l}|_{\mf{x_0,y_0,z_0}}=f_x\mf{x_0,y_0,z_0}\cos \alpha+f_y\mf{x_0,y_0,z_0}\cos \beta+f_z\mf{x_0,y_0,z_0}\cos \gamma$\p



\subsubsection{梯度}
$f\mf{x,y},D\mt{内连续偏导数},\forall P_0 \in D,\mb{grad} f\mf{x_0,y_0}=\Delta f\mf{x_0,y_0}=f_x|_{P_0}i+f_y|_{p_0}j,\mt{梯度,}\Delta=\da{}{x}i+\da{}{y}j\mt{向量微分算子}$\p
\qa

$\ma{f\mf{x,y}\mt{在}\mf{x_0,y_0}\mt{可微分,}\\ \mt{单位向量}e_l=\mf{\cos \alpha,\cos \beta}},\ma{\da{f}{l}|_{\mf{x_0,y_0}}&=f_x\mf{x_0,y_0}\cos \alpha+f_y\mf{x_0,y_0}\cos \beta\\
&=\mb{grad} f\mf{x_0,y_0}\cdot e_l\\
&=\mfa{\mb{grad} f\mf{x_0,y_0}}\cos \theta, \md{\theta =\mf{\widehat{\mb{grad} f\mf{x_0,y_0},e_l}}}
}$\p
\qa

$\ma{\mt{曲面}\,\sum z=f\mf{x,y},\mt{等值线}L\,f\mf{x,y}=0,P_0\mf{x_0,y_0},\\\mt{记}k=\sqrt{f_x^2\mf{x_0,t_0}+f_y^2\mf{x_0,y_0}},}
\Rightarrow \ma{\ma{ \mt{单位法向量}n&=\frac{1}{k}\mf{f_x|_p,f_y|_p}\\&=\frac{\Delta f\mf{x_0,y_0}}{\mfa{\Delta f\mf{x_0,y_0}}}}\\
    \Delta f\mf{x_0,y_0}=\mfa{\Delta f\mf{x_0,y_0}}n=\da{f}{n}n\\
    \mt{梯度=梯度的模\, 梯度的方向}
}$\p
\qa

$f\mf{x,y,z},G\mt{内连续偏导数},P_0\mf{x_0,y_0,z_0},\ma{\mb{grad}f=\Delta f=f_x|_{P_0}i+f_y|_{P_0}j+f_z|_{P_0}k,\\ 
\mt{等值面}f\mf{x,y,z}=c\\
\mt{等值面法线方向} n
}$\p

$\ma{\mt{空间区域}G \q \forall M \in G \rightarrow \mt{数量}f\mf{M},G\mt{内数量场},\\
\mt{空间区域}G \q \forall M \in G \rightarrow \mt{向量}f\mf{M},G\mt{内向量场},\\
if \q F\mf{M} \mt{是} f\mf{M}\mt{的梯度},then \q f\mf{M}\mt{势函数},F\mf{M}\mt{势场}
}$\p

$\frac{m}{r}\mt{引力势},\mb{grad}\frac{m}{r}\mt{引力场}$


\subsection{多元函数的极值及其求法}

\subsubsection{多元函数的极值及最大值最小值}
$D,z=f\mf{x,y},P_0\mf{x_0,y_0}\in D ,\ma{if \q \exists U\mf{P_0}\subset D,\forall \mf{x,y} \neq P_0,f\mf{x,y}<f\mf{x_0,y_0},then \q \mt{极大值}f\mf{x_0,y_0}\\
    if \q \exists U\mf{P_0}\subset D,\forall \mf{x,y} \neq P_0,f\mf{x,y}>f\mf{x_0,y_0},then \q \mt{极小值}f\mf{x_0,y_0}\\
    \mt{极大值,极小值统称极值}\\
}$\p

$if \q z=f\mf{x,y}\mt{在}\mf{x_0,y_0}\mt{具有偏导数,且在}\mf{x_0,y_0}\mt{点具有极值},then \q f_x\mf{x_0,y_0}=0,f_y\mf{x_0,y_0}=0$\p

$if \q f_x\mf{x_0,y_0}=0,f_y\mf{x_0,y_0}=0,then \q \mt{驻点}\mf{x_0,y_0}$

$if \q z=f\mf{x,y,z}\mt{在}\mf{x_0,y_0,z_0}\mt{具有偏导数,且在}\mf{x_0,y_0,z_0}\mt{点具有极值},then \q \ma{f_x\mf{x_0,y_0,z_0}=0,\\
f_y\mf{x_0,y_0,z_0}=0,\\
f_z\mf{x_0,y_0,z_0}=0
}$

$\ma{
    z=f\mf{x,y}\mt{在}\mf{x_0,y_0}\mt{的某邻域内连续,}\\ 
    \mt{且有一阶及二阶偏导数,又}f_x\mf{x_0,y_0}=0,f_y\mf{x_0,y_0}=0\\
    \mt{令}f_{xx}\mf{x_0,y_0}=A,f_{xy}\mf{x_0,y_0}=B,f_{yy}\mf{x_0,y_0}=C\\
}if \q \ma{
    AC-B^2>0,\mt{有极值},A<0\mt{极大值},A>0\mt{极小值}\\
    AC-B^2<0,\mt{无极值}\\
    AC-B^2=0,\mt{可能有极值}\\
}$


$\mt{驻点,偏导不存在点}$


\subsubsection{条件极值 $\q$ 拉格朗日数乘法}
无条件极值,条件极值;条件极值到无条件极值\p
$\ma{\mf{x_o,y_0}\mt{的某邻域内}z=f\mf{x,y},\varphi\mf{x,y},\\
\mt{有连续一阶导数},\varphi\mf{x,y}=0\mf{\mt{条件}}\rightarrow y=\varphi\mf{x},\\
z=f\mf{x,\varphi\mf{x}}\mf{\mt{函数}}}\Rightarrow \ma{
    \frac{dz}{dx}|_{x=x_0}=0\\
    f_x\mf{x_0,y_0}+f_y\mf{x_0,y_0}\frac{dy}{dx}|_{x=x_0}=0\\
    f_x\mf{x_0,y_0}-f_y\mf{x_0,y_0}\frac{\varphi_x\mf{x_0,y_0}}{\varphi_y\mf{x_0,y_0}}=0\\
    \mt{记}\frac{f_y\mf{x_0,y_0}}{\varphi_y\mf{x_0,y_0}}=-\lambda\\
    f_y\mf{x_0,y_0}-f_y\mf{x_0,y_0}\frac{\lambda}{\lambda}=0\\
    \fcz{f_x\mf{x_0,y_0}+\lambda\varphi_x\mf{x_0,y_0}=0\\f_y\mf{x_0,y_0}+\lambda\varphi_y\mf{x_0,y_0}=0\\ \varphi\mf{x_0,y_0}=0}
 
}$

$L\mf{x,y}=f\mf{x,y}+\lambda\varphi\mf{x,y}\Rightarrow \fcz{
    L_x\mf{x_0,y_0}=0\\
    L_y\mf{x_0,y_0}=0\\
    \varphi\mf{x_0,y_0}=0
}\ma{\mt{拉格朗日函数}L,\mt{拉格朗日乘子}\lambda\\
    \mt{函数}f\mf{x,y},\mt{条件}\varphi\mf{x,y}=0
}$


$L\mf{x,y,z,t}=f\mf{x,y,z,t}+\lambda\varphi\mf{x,y,z,t}+\mu\psi\mf{x,y,z,t} \Rightarrow \fcz{
    L_x\mf{x_0,y_0,z_0,t_0}=0\\
    L_y\mf{x_0,y_0,z_0,t_0}=0\\
    L_z\mf{x_0,y_0,z_0,t_0}=0\\
    L_t\mf{x_0,y_0,z_0,t_0}=0\\
    \varphi\mf{x_0,y_0,z_0,t_0}=0\\
    \psi\mf{x_0,y_0,z_0,t_0}=0
}\ma{\mt{拉格朗日函数}L,\\
\mt{拉格朗日乘子}\lambda,\mu \\
    \mt{函数}f\mf{x,y,z,t},\\
    \mt{条件}\varphi\mf{x,y,z,t}=0,\\
    \psi\mf{x,y,z,t}=0
}$


\subsubsection{二元函数的泰勒公式}
$\ma{z=f\mf{x,y}\mt{在点}P_0\mf{x_0,y_0}\mt{某邻域内连续,}\\
\mt{有n+1阶连续偏导数,}\\
\mf{x_0+h,y_0+k} \in U\mf{P_0,\delta}}\Rightarrow \ma{f\mf{x_0+h,y_0+k}=f\mf{x_0,y_0}+\mf{h\da{}{x}+k\da{}{y}}f\mf{x_0,y_0}\\
    \frac{1}{2!}\mf{h\da{}{x}+k\da{}{y}}^2f\mf{x_0,y_0}+\dots+\frac{1}{n!}\mf{h\da{}{x}+k\da{}{y}}^mf\mf{x_0,y_0}\\
    \frac{1}{\mf{n+1}!}\mf{h\da{}{x}+k\da{}{y}}^{n+1}f\mf{x_0+\theta h,y_0+\theta k} \q \mf{0<\theta<1}
}$


$\mt{证明:}\ma{\Phi\mf{t}=f\mf{x_0+h,y_0+k} \, \mf{0\leqslant t \leqslant 1}\\
\Phi^{\mf{n}}\mf{t}=\mf{h\da{}{x}+k\da{}{y}}^nf\mf{x_0+ht,y_0+kt}\\
\Phi\mf{t}=\Phi\mf{0}+\Phi^{'}\mf{0}+\frac{1}{2}\Phi^{''}\mf{0}+\cdots+\frac{1}{n!}\Phi^{\mf{n}}\mf{0}+\frac{1}{\mf{n+1}!}\Phi^{\mf{n+1}}\mf{\theta t} \,\mf{0<\theta<1}\mf{0处展开}\\
\Phi\mf{1}=\Phi\mf{0}+\Phi^{'}\mf{0}+\frac{1}{2}\Phi^{''}\mf{0}+\cdots+\frac{1}{n!}\Phi^{\mf{n}}\mf{0}+\frac{1}{\mf{n+1}!}\Phi^{\mf{n+1}}\mf{\theta } \,\mf{0<\theta<1}\mf{较远点}
}$
\qa

$n=0\mt{时}f\mf{x_0+h,y_0+k}-f\mf{x_0,y_0}=hf_x\mf{x_0+\theta h,y_0+\theta k}+kf_y\mf{x_0+\theta h,y_0+\theta k}$
\qa 

$f\mf{x,y},\mf{x,y}\in D,f_x\mf{x,y} \equiv 0,f_y\mf{x,y} \equiv 0,f\mf{x,y}= c$\p



\subsubsection{极值充分证明}

$\ma{
    \Delta f 
    &=f\mf{x,y}-f\mf{x_0,y_0}\\
    &=\mfc{h\da{}{x}+k\da{}{y}}f\mf{x_0,y_0}+\frac{1}{2}\mfc{h\da{}{x}+k\da{}{y}}^2f\mf{x_0+\theta h,y_0+\theta k}\\
    &f_x\mf{x_0,y_0}=0,f_y\mf{x_0,y_0}=0\\
    &=\frac{1}{2}\mfc{h\da{}{x}+k\da{}{y}}^2f\mf{x_0+\theta h,y_0+\theta k}\\
    &=\frac{1}{2}\mf{h^2f_{xx}+2hkf_{xy}+k^2f_{yy}}\\
    &=\frac{1}{2f_{xx}}\mf{h^2f_{xx}^2+2hkf_{xy}f_{xx}+k^2f_{xy}^2-k^2f_{xy}^2+k^2f_{yy}f_{xx}}\\
    &=\frac{1}{2f_{xx}}\mfc{ \mf{hf_{xx}+kf_{xy}}^2+k^2\mf{f_{xx}f_{yy}-f_{xy}^2}}\\
    &A=f_{xx},C=f_{yy},B=f_{xy}\\
    &\ma{AC-B^2>0\\ \mt{有}}\fcz{A>0,f\mf{x,y}>f\mf{x_0,y_0},\mt{极小值}\\
        A<0,f\mf{x,y}<f\mf{x_0,y_0},\mt{极大值}\\}\\
    &\ma{AC-B^2<0\\ \mt{无}}\fcz{
        A=C=0,k=h,\Delta f=f_{x,y}\\
        A=C=0,k=-h,\Delta f=-f_{x,y}\\
        A\neq C=0,k=0,\Delta f=\frac{1}{2}h^2f_{xx}\\
        A\neq C=0,h=-f_{xy}s,k=f_{xx}s,
        \ma{\Delta f &=\frac{1}{2}\mfb{f_{xy}^2f_{xx}-2f_{xy}^2f_{xx}+f_{xx}^2f_{yy}}\\
            &=\frac{1}{2}f_{xx}\mfb{f_{xy}^2-2f_{xy}^2+f_{xx}f_{yy}}\\
            &=\frac{1}{2}f_{xx}\mfb{ -f_{xy}^2 }\\
         }
    }\\
    &\ma{AC-B^2=0\\ \mt{有可能}}\fcz{
        f\mf{x,y}=x^2+y^4\\
        g\mf{x,y}=x^2+y^3
    }
}$


\subsubsection{最小二乘法}

$\ma{
y_i \mt{数据} \Rightarrow f\mf{t}=at+b \mt{线性经验公式}\\
M=\sum\mfc{y_i-f\mf{t_i}}^2=\sum\mfc{y_i- \mf{at_i+b }}^2\mt{偏差平方和,}\\
\mt{M最小为条件选择常数a,b的方法,最小二乘法}\\
\sqrt{M}\mt{均方误差}\\
\mt{非线性到线性}
M_{min}\mf{a,b} \Rightarrow  \ma { 
\fcz{
        M_{a}\mf{a,b}=0\\
        M_{b}\mf{a,b}=0
} \\   
\fcz{
    \da{M}{a}=-2t_i\sum\mfc{y_i- \mf{at_i+b} }=0\\
    \da{M}{b}=-2\sum\mfc{y_i- \mf{at_i+b} }=0
} \\   
\fcz{
    \sum t_iy_i-a\sum {t_i}^2-b\sum y_i=0\\
    \sum y_i-a\sum t_iy_i -bn=0
}\\
}\Rightarrow \fcz{
    a=\frac{\hls{
         \sum t_iy_i &  \sum y_i\\
         \sum y_i  & n
    }}{\hls{
         \sum {t_i}^2 &  \sum y_i\\
         \sum t_iy_i  & n
    }} \\
    b=\frac{\hls{
         \sum {t_i}^2 &  \sum t_iy_i\\
         \sum t_iy_i  & \sum y_i
    }}{\hls{
         \sum {t_i}^2 &  \sum y_i\\
         \sum t_iy_i  & n
    }}
}
}$



\newpage
\section{重积分}

\subsection{二重积分的概念与性质}

\subsubsection{二重积分的概念}

\mb{曲顶柱体的体积}\p
$V=\lim_{\lambda \rightarrow 0}\sum f\mf{\xi_i,\eta_i}\Delta \sigma _i$\p

\mb{平面薄片的质量}\p
$m=\lim_{\lambda \rightarrow 0}\sum \mu\mf{\xi_i,\eta_i}\Delta \sigma _i$

$\ma{\mt{闭区域}D\mt{有界函数}\\
\mt{分割}D,\Delta \sigma_1,\Delta \sigma_2,\dots,\Delta \sigma_n
}\Rightarrow   \iint_D  f\mf{x,y}d\sigma=\lim_{\lambda \rightarrow 0}f\mf{\xi_i,\eta_i}\Delta \sigma\mt{二重积分}$\p

$\Delta \sigma_i=\Delta x_j \cdot y_k,d \sigma \rightarrow dxdy, \iint_D  f\mf{x,y}dxdy$

\subsubsection{二重积分的性质}
$\ma{g\mf{x,y},g\\
    f\mf{x,y},f \\
    }
    \Rightarrow \ma{
    \iint_D \mfc{\alpha f +\beta g }d \sigma=\alpha \iint_D  f  d \sigma +\beta \iint_D  g d \sigma\\
    \iint_D f d \sigma=\iint_{D_1} f d \sigma+\iint_{D_2} f d \sigma\\
    \iint_D 1 \cdot d \sigma=\iint_D  \cdot d \sigma=\sigma\\
    \iint_D  f d \sigma \leqslant \iint_D  g d \sigma ,f \leqslant g\\
    m\sigma \leqslant  \iint_D  f d \sigma \leqslant M\sigma ,m\leqslant f \leqslant M \\
    \iint_D  f d \sigma=f\mf{\xi,\eta}\sigma \q \mf{\xi,\eta} \in D

}$

\subsection{二重积分的计算法}

\subsubsection{直角坐标计算二重积分}
$x\mt{型}\fcz{D:\varphi_1\mf{x}\leqslant y \leqslant\varphi_1\mf{x},a \leqslant x \leqslant b \\
A\mf{x_0}=\int_{\varphi_1\mf{x_0}}^{\varphi_2\mf{x_0}}f\mf{x_0,y}dy\\
V=\int_a^b A\mf{x}dx=\int_a^b \mfc{\int_{\varphi_1\mf{x}}^{\varphi_2\mf{x}}f\mf{x,y}dy}  dx =\int_a^b dx\int_{\varphi_1\mf{x}}^{\varphi_2\mf{x}}f\mf{x,y}dy=\iint_D f\mf{x,y}d\sigma\\
}$\p

$y\mt{型}\fcz{D:\psi_1\mf{y}\leqslant x \leqslant\psi_1\mf{y},a \leqslant y \leqslant b \\
A\mf{y_0}=\int_{\psi_1\mf{y_0}}^{\psi_2\mf{y_0}}f\mf{x,y_0}dx\\
V=\int_a^b A\mf{y}dy=\int_a^b \mfc{\int_{\psi_1\mf{y}}^{\psi_2\mf{y}}f\mf{x,y}dx}  dy =\int_a^b dy\int_{\psi_1\mf{y}}^{\psi_2\mf{y}}f\mf{x,y}dx=\iint_D f\mf{x,y}d\sigma\\
}$

非x非y型,分割积分面\p

\subsubsection{极坐标计算二重积分}
$\ma{S_{\mt{扇}}=\frac{r^2\theta}{2}\\
\Delta \sigma = \frac{1}{2}\mfc{\mf{r+\Delta r}^2-r^2}\Delta \theta=\frac{1}{2}\mfc{ {\Delta r}^2 +2r\Delta r}\Delta \theta=
\frac{1}{2}\mfc{ {\Delta r} +2r } \Delta r \Delta \theta=\overline{r}\Delta r \Delta \theta \\
\lim_{\lambda \rightarrow 0}\sum f\mf{\xi_i,\eta_i}\Delta \sigma _i=\lim_{\lambda \rightarrow 0 }\sum f\mf{\overline{r}\cos \overline{\theta},\overline{r} \sin \overline{\theta}}\overline{r}\Delta r \Delta \theta\\
\iint_D f\mf{x,y}d \sigma=\iint_D f\mf{x,y}dxdy=\iint_D f\mf{r\cos \theta,r \sin \theta}rdrd\theta\\
\iint_D f\mf{r\cos \theta,r \sin \theta}rdrd\theta=\int_a^b \mfc{\int_{\varphi_1\mf{\theta}}^{\varphi_2\mf{\theta}} f\mf{r\cos \theta,r \sin \theta}rdr} d\theta\\
\sigma=\iint_Drdrd\theta=\frac{1}{2}\int_a^b\mfc{{\varphi_2\mf{\theta}}^2-{\varphi_1\mf{\theta}}^2 }d\theta = 
\frac{1}{2}\int_a^b{\varphi_2\mf{\theta}}^2  d\theta\\
}$

 
\subsubsection{二重积分换元法}

$\ma{f\mf{x,y},D\mt{内连续},\\
    \mt{变换}T: x=x\mf{u,v},y=y\mf{u,v}\\
    uOv\mt{面}D^{'}\rightarrow xOy\mt{面}D,\mt{一对一}\\
    x\mf{u,v},y\mf{u,v}\mt{在}D^{'}\mt{一阶连续偏导},\\
    D^{'}\mt{上雅可比式,}J\mf{u,v}=\da{\mf{x,y}}{\mf{u,v}}\neq 0,\\
}\Rightarrow \iint_Df\mf{x,y}dxdy=\iint_{D^{'}}f\mf{x\mf{u,v},y\mf{u,v}}\mfa{J\mf{u,v}}dudv
$

$\ma{c^2=a^2+b^2-2ab\cos c\\
    \fcz{
        M_1:x_1=x\mf{u,v},y_1=y\mf{u,v}\\
        M_2:x_2=x\mf{u+h,v}=x_1+x_u\mf{u,v}h+o\mf{h}\\
            y_2=y\mf{u+h,v}=y_1+y_u\mf{u,v}h+o\mf{h}\\
        M_4:x_3=x\mf{u,v+h}=x_1+x_v\mf{u,v}h+o\mf{h}\\
            y_3=y\mf{u,v+h}=y_1+y_v\mf{u,v}h+o\mf{h}\\
    }\\}$\p
$
\ma{
    s_{\mt{三角形}}&=\frac{1}{2}\mfa{\ma{
        x_1 &y_1 &1 \\
        x_2 &y_2 &1 \\
        x_3 &y_3 &1 \\
    }}=\frac{1}{2}\mfa{\ma{
        x_2-x_1 &y_2-y_1  \\
        x_3-x_1 &y_3-y_1  \\
    }}\\
    &=\frac{1}{2}\mfa{\ma{
        x_u\mf{u,v}h  &y_u\mf{u,v}h\\
        x_v\mf{u,v}h  &y_v\mf{u,v}h\\
    }}=\frac{1}{2}h^2\mfa{\ma{
        x_u\mf{u,v}  &y_u\mf{u,v}\\
        x_v\mf{u,v}  &y_v\mf{u,v}\\
    } 
    }=\frac{1}{2}h^2\mfa{\ma{
        x_u\mf{u,v}  &x_v\mf{u,v}\\
        y_u\mf{u,v}  &y_v\mf{u,v}\\
    }}=\frac{1}{2}h^2 \da{\mf{x,y}}{\mf{u,v}}\\

}
$

\subsection{三重积分}

\subsubsection{三重积分的概念}
$\ma{f\mf{x,y,z},\mt{有界闭区域}\Omega ,\\
\mt{分割成} \Delta v_1,\Delta v_2,\cdots,\Delta v_n,\\}\Rightarrow
\iiint_D f\mf{x,y,z}dv=\iiint_D f\mf{x,y,z}dxdydz=\lim_{\lambda \rightarrow 0}\sum f\mf{\xi_i,\eta_i,\zeta_i }\Delta v_i
$
\subsubsection{三重积分的计算}
\mb{利用直角坐标系计算}\p
$\ma{F\mf{x_0,y_0}=\int_{z_1\mf{x_0,y_0}}^{z_2\mf{x_0,y_0}}f\mf{x_0,y_0,z}dz\\
\iiint_{\Omega}f\mf{x,y,z}dv\ma{
=\iint_{D_{xy}} F\mf{x,y}d\sigma=\iint_{D_{xy}}\mfc{\int_{z_1\mf{x,y}}^{z_2\mf{x,y}}f\mf{x,y,z}dz}d\sigma
=\int_a^b dx \int_{y_1\mf{x}}^{y_2\mf{x}} dy \int_{z_1\mf{x,y}}^{z_2\mf{x,y}} f\mf{x,y,z}dz\\
=\int_{c_1}^{c_2}dz\iint_{D_z}f\mf{x,y,z}dxdy
}
}$

\mb{利用柱坐标系计算}\p
$\ma{\fcz{
    x=r\cos \theta\\
    y=r\sin \theta\\
    z=z
}\\
dv=rdrd\theta dz\\
}\Rightarrow \iiint_{\Omega}f\mf{x,y,z}dxdydz=\iiint_{\Omega}f\mf{r\cos \theta,r\sin \theta,z}rdrd\theta dz$



\mb{利用球面坐标计算}\p
$\ma{
    \fcz{
        x=OP \cos \theta =r \sin \varphi \cos \theta\\
        y=OP \sin \theta =r \sin \varphi \sin \theta\\ 
        z=r \cos \varphi
    } \\
    \varphi \mt{,Z轴}; \theta \mt{,X轴}\\
    dv=r^2\sin \varphi dr d \varphi d\theta
}\iiint_{\Omega}f\mf{x,y,z}dxdydz=\iiint_{\Omega}f\mf{r \sin \varphi \cos \theta,r \sin \varphi \sin \theta,r \cos \varphi}r^2\sin \varphi dr d \varphi d\theta
$


\subsection{重积分的应用}
$\mb{曲面的面积}$\p
$\ma{
    \mt{曲面} S: z=f\mf{x,y}\\
    \mfa{\mf{f_x,f_y,-1}\cdot\mf{0,0,1}}=1=\sqrt{{f_x}^2+{f_y}^2+{-1}^2}\cdot1\cdot\cos\theta,\\
    \cos \theta=\frac{1}{\sqrt{{f_x}^2+{f_y}^2+\mf{-1}^2}}\\
    \frac{d\sigma}{dA}=\cos \theta ,\frac{\sum \sigma_k}{\sum A_k}=\cos \theta,dA=\frac{1}{\cos \theta}d\sigma
}A=\iint_{D}\sqrt{{f_x}^2+{f_y}^2+1 }d\sigma$\p

$\mb{*利用曲面的参数方程求曲面的面积}$\p

$\fcz{
    x=x\mf{u,v}\\
    y=y\mf{u,v}\\
    z=z\mf{u,v}\\
}$















\newpage
\section*{tikz test}

\tikz\draw[<->](0pt,0pt)--(3,0);
\tikz\draw[o-stealth](0pt,0pt)--(3,0);
 
\tp{
    \tpo{a}{(0,0)};
    \tpo{b}{(3,0)};
    \draw[<->](a)--(b);
}


\tp{
\draw[style= dashed]  (2,0.5) circle (0.5);
\draw[fill=green]  (1,1) ellipse (0.5 and 1); 
\draw[fill=blue]  (0,0) rectangle (1 , 1); 
}

\tp{
    \draw(0,0)--(90:1)arc(90:390:1)--cycle;
    \draw(60:5pt)-- +(30:1)arc(30:90:1)-- cycle;
}

\tp{
    \draw(0,0)--(2,2);
    \draw(2,0)--(0,2);
    \draw \tip{0}{0}{2}{2}{2}{0}{0}{2} circle (.3)
    % scale rotate xslant
    [shift={(.45,.45)}] \tip{0}{0}{2}{2}{2}{0}{0}{2} circle (.3)
    [shift={(.45,.45)}] \tip{0}{0}{2}{2}{2}{0}{0}{2} circle (.3);
}
\tp{
    \coordinate [label=0:$a$] (a) at (0,0);
    \coordinate [label=0:$b$] (b) at (2.5,3);
    \coordinate [label=0:$d$] (d) at (0,-1);
    \coordinate [label=0:$e$] (e) at (2,3);
    \coordinate [label=0:$c$] (c) at (intersection of a--b and d--e);
    \draw (a)--(b)  (d)--(e);
    \foreach \p in {a,b,c,d,e} \fill [opacity = 0.75] (\p) circle (2pt);


}

\tikz\path[fill=green!30] (1,1)--(2,2)--(3,1)--cycle;


\newpage


\end{document}




