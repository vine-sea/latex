\documentclass[UTF8]{ctexart}
 
\usepackage{anyfontsize}  %去除ctex字體報錯
\usepackage{amsmath}
\usepackage{amssymb} 
\usepackage{extarrows}
\usepackage{titlesec}
\usepackage{titletoc}
\usepackage{tikz}
\usetikzlibrary{arrows,backgrounds}
\newcommand{\Rmnum}[1]{\uppercase\expandafter{\romannumeral #1}} 
\newcommand{\mR}[1]{\uppercase\expandafter{\romannumeral #1}} 

\newcommand{\mt}[1]{\text{#1}}
\newcommand{\mb}[1]{\textbf{#1}}
\newcommand{\md}[1]{\displaystyle{#1}}
\newcommand{\mf}[1]{\left( #1\right)}
\newcommand{\mfa}[1]{\left| #1\right|}
\newcommand{\mfb}[1]{\left\{ #1\right\}}
\newcommand{\mfc}[1]{\left[ #1 \right]}
\newcommand{\q}{\quad}
\newcommand{\qa}{\vspace{12 pt}}

\newcommand{\mh}[2]{\overset{#2}{#1}}
\newcommand{\mha}[1]{\overrightarrow{#1}}
\newcommand{\p}{\par}
\newcommand{\ma}[1]{\begin{array}{llll} #1 \end{array}}
\newcommand{\tp}[1]{\begin{tikzpicture}  #1 \end{tikzpicture}}
\newcommand{\tpa}[1]{
    \begin{center}
        \begin{tikzpicture}  
            % [scale=1 ,show background rectangle] 
        
            #1 
            \end{tikzpicture}
    \end{center}
}
\newcommand{\tip}[8]{(intersection of #1,#2--#3,#4 and #5,#6--#7,#8)}

\newcommand{\tpo}[2]{\coordinate  %[label=0:$ {#1} $]
 (#1) at #2; }

\newcommand{\da}[2]{\frac{\partial #1}{\partial #2}}
\newcommand{\db}[2]{\frac{d #1}{d #2}}
\newcommand{\fcz}[1] {
    \left\{
        \begin{array}{llll} #1 \end{array}
    \right.
}

\newcommand{\hls}[1] {
    \left|
        \begin{array}{llll} #1 \end{array}
    \right|
}

\newcommand{\ba}[1]{\overline{#1}}
\newcommand{\meq}[2]{\xlongequal[#2]{#1}}
\newcommand{\mseq}{\approx }


\title{This is Start}
\author{Vine}
\date{\today}

\usepackage{geometry}
\geometry{papersize={21cm,29.7cm}}
\geometry{left=1cm,right=1cm,top=2cm,bottom=2cm}

\usepackage{fancyhdr}
\pagestyle{fancy}

\lhead{Vine}
\chead{}
\rhead{}

\lfoot{}
\cfoot{\thepage}
\rfoot{}
\renewcommand{\headrulewidth}{0.1pt}
\renewcommand{\headwidth}{\textwidth}
\renewcommand{\footrulewidth}{0pt}

\usepackage{setspace}
\onehalfspacing
\usepackage{indentfirst}


\begin{document}
\setlength{\headheight}{15pt}
% \maketitle %和頁眉衝突
\newpage
\tableofcontents{}
\newpage
\section{$DT\Rmnum{2}\mf{A}$型帶式輸送機產品系列}
\subsection{適用範圍}
\subsection{產品規格}
\subsection{整機結構,部件名稱及代碼}
\subsection{整機典型配置}
\subsection{圖紙編號規則}



\newpage
\section{整機設計}
\subsection{散狀物料的特性}
\subsection{帶速的選擇}
\subsection{總體佈置(側型)設計}
\subsection{滾筒匹配}
\subsection{托輥間距}
 


\newpage
\section{設計計算}
\subsection{計算標準,符號和單位}
\subsection{原始數據及工作條件}
\subsection{輸送能力和輸送帶寬度}
\subsection{圓周驅動力}
\subsection{輸送帶張力}
\subsection{傳動滾筒軸功率}
\subsection{逆止力計算和逆止器選擇}
\subsection{電動機功率和驅動裝置組合}
\subsection{輸送帶選擇計算}
\subsection{拉緊參數計算}
\subsection{凹凸弧段尺寸}
\subsection{啟動和制動}
\subsection{雙滾筒驅動計算}
\subsection{下運帶式輸送機計算}
\subsection{典型計算式例}



\newpage
\section{部件選型}
\subsection{輸送帶}
\subsection{驅動裝置}
\subsection{逆止器}
\subsection{傳動滾筒}
\subsection{改向滾筒}
\subsection{托輥}
\subsection{拉緊裝置}
\subsection{清掃器}
\subsection{機架}
\subsection{頭部漏斗}
\subsection{導料槽}
\subsection{卸料裝置}





\newpage
\section{輸送機系統設計}
\subsection{輸送能力和計算依據}
\subsection{負荷啟動和超載}
\subsection{部件選型的一致性原則}
\subsection{系統控制}
\subsection{輔助和配套設備配置方式}



\newpage
\section{主要部件型譜}
\subsection{傳動滾筒}
\subsection{改向滾筒}
\subsection{承載托輥}
\subsection{回程托輥}
\subsection{托輥棍子}
\subsection{垂直重錘拉緊裝置}
\subsection{車式重錘拉緊裝置}
\subsection{螺旋拉緊裝置}
\subsection{電動絞車拉緊裝置}
\subsection{清掃器}
 




\newpage
\section{驅動裝置型譜}

\subsection{驅動裝置的組成說明}
\subsection{Y-ZLY-ZSY(Y-DBY/DCY)驅動裝置}
\subsection{Y-ZLY-ZSY驅動裝置}
\subsection{Y-DBY/DCY驅動裝置}
\subsection{驅動裝置和傳動滾筒的組合}
\subsection{驅動裝置架}
\subsection{梅花聯軸器護罩}
\subsection{液力耦合器護罩}




\newpage
\section{電動滾筒}
\subsection{概述}
\subsection{內置式電動滾筒}
\subsection{外置式電動滾筒相關信息}





\newpage
\section{結構件選型}
\subsection{傳動滾筒頭架}
\subsection{角型改向滾筒頭架}
\subsection{中部傳動滾筒支架}
\subsection{中部改向滾筒架}
\subsection{改向滾筒尾架}
\subsection{垂直拉緊裝置}
\subsection{車式拉緊裝置}
\subsection{螺旋拉緊裝置尾架}
\subsection{中間架}
\subsection{尾架}
\subsection{導料槽}
\subsection{頭部漏斗}


\newpage
\section{輔助裝置型譜}
\subsection{壓輪}
\subsection{輸送帶水洗裝置}
\subsection{輸送帶除水裝置}
\subsection{輸送機罩}
\subsection{犁式卸料器}
\subsection{卸料車}
\subsection{重型卸料車}
\subsection{重型卸料車專用中部支架}
\subsection{可逆配倉帶式輸送機}
\subsection{重型可逆配倉帶式輸送機}
 



\newpage
\section{安全規範與防護技術}
\subsection{帶式輸送機安全規範的一般規定}
\subsection{機電設備的防爆}
\subsection{易燃部件的阻燃要求}
\subsection{帶式輸送機擠壓部分的防護}
\subsection{帶式輸送機人行通道}





\newpage
\section{相關設備和設施}
\subsection{輸送機集中潤滑系統}
\subsection{輸送機除塵裝置}
\subsection{移動供電裝置}
\subsection{輸送機系統控制檢測元件}
\subsection{輸送機通廊}




\newpage
\section{計算機輔助設計}
\subsection{概述}
\subsection{使用範圍-常用側型}
\subsection{功能簡要說明}
\subsection{用戶手冊示例}

\newpage
\section{其他類型輸送機部件(一)}
\subsection{概述}
\subsection{傳動裝置}
\subsection{改向壓輪}
\subsection{清掃器}
\subsection{卸料器}
\subsection{頭架}
\subsection{螺旋拉緊裝置}
\subsection{中部改向滾筒支架}
\subsection{垂直拉緊裝置}
\subsection{增面滾筒支架}
\subsection{中間架}
\subsection{中間支腿及斜撐}
\subsection{頭罩}
\subsection{尾輪防護罩}
\subsection{犁式卸料器除塵罩}
\subsection{卸料漏斗}
\subsection{導料槽}
\subsection{磁選單元}

\newpage
\section{其他類型輸送機部件(二)}
\subsection{概述}
\subsection{頭部支架}
\subsection{尾部支架}
\subsection{車式拉緊裝置尾部支架}
\subsection{中間支架及支腿}
\subsection{頭部護罩}
\subsection{頭部漏斗}
\subsection{導料槽}
\subsection{車式拉緊裝置}
\subsection{Y-ZSY驅動裝置}

\newpage
\section{輸送帶及接頭產品}
\subsection{雙箭橡膠}
\subsection{華夏橡膠}
\subsection{富大橡膠}
\subsection{歐耐橡膠}
\subsection{輸送帶接頭}
\subsection{DPL型帶接頭}
\subsection{膠帶修補器}
\subsection{膠帶剝皮機}




\newpage
\section{驅動裝置標準部件產品資料}
\subsection{電機}
\subsection{減速器}
\subsection{聯軸器}
\subsection{液力耦合器}
\subsection{鋼球耦合器}
\subsection{制動器}
\subsection{逆止器}




\newpage
\section{帶式輸送機配套產品資料}
\subsection{KA系列帶式輸送機托輥軸承}
\subsection{賬套}
\subsection{托輥沖壓軸承座}
\subsection{托輥密封圈(尼龍)}
\subsection{清掃器}
\subsection{特型托輥和緩衝床}
\subsection{輸送機安全保護裝置}
\subsection{電磁分離器}
\subsection{襯板-超耐磨陶瓷復合襯板}
\subsection{DTY(B)型系列電液推桿}
\subsection{DYN-$\Rmnum{4}$型電液動犁式卸料器}
\subsection{DZYLJ(B)型帶式輸送機自控液壓拉緊/糾偏裝置}
\subsection{帶式輸送機防雨罩}
\subsection{膠帶修補器}




\newpage
\section{附錄}


\end{document}




